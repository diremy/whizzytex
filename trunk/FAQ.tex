\documentclass{article}

\usepackage {color}
\usepackage {hyperref}
\usepackage {hevea}

\title {\textcolor{blue}{WhizzyTeX}  Frequently Asked Questions}
\author {Didier R{\'{e}}my}
\date {Last Modified: \today}

\let \lst \verb

\input {version}

\begin{document}

\maketitle

\tableofcontents

\section {Where to find it}

The URL is \url{http://pauillac.inria.fr/whizzytex}

The last major release is {\release}
and the stable version is {\version}

The last version of the FAQ can be found at 
\url{http://pauillac.inria.fr/whizzytex/FAQ.html}


\section {Known problems}

Fixed refers to last stable version.
\def \FIXED {\textit{Fixed in version $\ge$ {\version}}}

\begin {enumerate}

\item
Its has been reported with versions that sometimes the cursor jumps to the
end of the buffer while typing fast, and the mark is left a few characters
before where the point was when the cursor jumped.

The problem should be \FIXED. 

Otherwise, an (imperfect) workaround is to reduce the slicing speed
\emph{e.g.}  by the follow setting (you may need to adjust the value):
\begin{verbatim}
        (setq whizzy-load-factor 0.1)
\end{verbatim}


\item Its does not work with the  \lst"x-symbol" package.

The \lst"x-symbol" package may now work together with WhizzyTeX, provided
you have a version of {\tt x-symbol} greater than v4.3.3.  However, the
option Page-to-point may still not work (under \lst"x-symbol", the cursor
would be inserted at the end of the buffer instead of at the
current-position).  Then, you may use the following alternative.

Alternatively, with (X)Emacs version 21 and above, you may set
\lst"whizzy-write-annotate" to nil to tell WhizzyTeX not to use
\lst"write-region-annotation-functions". The slice will be write slower, but
in a way so that the \lst"x-symbol" package should always be compatible with 
WhizzyTeX. 

You may also wish to set \lst"'x-symbol-auto-conversion-method" to
\lst"'fast".

\item How can a click in the \lst"advi" window reposition the point
in the emacs window?

First, you should be using the \lst"advi" previewer.  Then, check the
documentation of \lst"advi", since bindings may depend on versions. This is
likely to be \lst"shift-left-mouse" or \lst"left-mouse".

There might also be a short-cut that temporarily rebinds
\lst"shift-xxx-mouse" to \lst"xxx-mouse". This can be toggled by typing
\lst"e"   in the \lst"advi" or by calling \lst"advi" with the option
\lst"-edit". 
%
Note that \lst"shift-midddle-click" and \lst"shift-right-click" (or their
shortcut versions) are used for \lst"move" and \lst"resize" commands when
WhizzyEditting.


\end {enumerate}


\section {Using WhizzyTeX with platex}

WhizzyTeX has been designed to work with standard LaTeX.
However, it should also work  with other implementations of LaTeX, as long
as they allow the creation of new formats. 

As an example, to make it work with \lst"platex" you can edit 
the script \lst"whizzytex" and change some findings as follows
(in the experimental version): 
\begin{verbatim}
INITEX=iniptex
LATEX=platex
FORMAT=platex
\end{verbatim}

\end{document}
