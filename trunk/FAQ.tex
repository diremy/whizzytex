\documentclass{article}

\usepackage {color}
\usepackage {hyperref}
\usepackage {hevea}
\def \WhizzyTeX{Whizzy\TeX}

\title {\textcolor{blue}{WhizzyTeX}  Frequently Asked Questions}
\author {Didier R{\'{e}}my}
\date {Last Modified: \today}

\let \lst \verb

\input {doc/version}

\begin{document}

\maketitle

\tableofcontents

\section {Where to find it}

The URL is \url{http://cristal.inria.fr/whizzytex/}

The last major release is {\release}
and the stable version is {\version}

The last version of the FAQ can be found at 
\url{http://cristal.inria.fr/whizzytex/FAQ.html}

\hypertarget{isitabug}{}
\section {Is this a Bug?}

Before reporting a bug, have you checked that the source file correctly
compiles and previews with the \emph{latex}, \emph{previewer}
and \emph{options} used in {\WhizzyTeX}?
(In recent versions of {\WhizzyTeX} you may see the previewer's command from 
\texttt{Emacs} using the menu \texttt{View log...} and selecting
\texttt{command}.) 

You may also check the latex log files (\texttt{format}, and
\texttt{latex}, \texttt{slice}). In particular, \texttt{format} will show
you the version of \texttt{whizzytex.sty} that has been loaded. Loading an
old version will likely not work! (You may also check the Emacs version of
whizzytex (from the \texttt{Help} entry of \texttt{Whizzy} menu). 

Since the latex \texttt{color} package is implicitly loaded, you 
should also check the compatibility with this package by 
by explicitly including \verb"\usepackage{color}" right before the
\verb"\begin{document}" (because whizzytex would load this package last) 
and checking that it still compiles with \emph{latex}.

In \texttt{advi} mode (when option \texttt{-advi} is used), the
\texttt{advi} package is also implicitly loaded. So you should then also try
to latex your source with the package \texttt{advi} explicitly loaded before
all other packages, since some errors may be indirectly due to this
\texttt{advi} and not to {\WhizzyTeX} itself---then send your bug report
to \href{http://cristal.inria.fr/advi/}{Active-DVI}.

Also, do not forget to check your version number before repporting a bug. 
See the value of variable \texttt{whizzytex-mode} in Emacs manually or
through the Emacs on-line documentation of \texttt{whizzytex-mode}. 
This version number should be identical to the one of
\texttt{whizzytex.sty}, which is displayed on the welcoming page when
starting WhizzyTeX, and also  to the one of the script \texttt{whizzytex},
which you can check with the shell command \texttt{whizzytex -version}.
If these three version numbers do not match, you have a broken
installation, maybe a file of old version takes priority, which you can
solve by deleled the old file or adjusting the corresponding PATH variable.


\section {Known problems}

Fixed refers to last stable version.
\def \FIXED {\textit{Fixed in versions $\ge$ {\version}}}

\begin {enumerate}

\item
{\WhizzyTeX} uses hard links and hence does not work on file systems that do
not support them, such as VFAT.

\item
It has been reported with old versions that sometimes the cursor jumps to the
end of the buffer while typing fast, and the mark is left a few characters
before where the point was when the cursor jumped.

The problem should be \FIXED. 

Otherwise, an (imperfect) workaround is to reduce the slicing speed
\emph{e.g.}  by the follow setting (you may need to adjust the value):
\begin{verbatim}
        (setq whizzy-load-factor 0.1)
\end{verbatim}


\item Its does not work with the  \lst"x-symbol" package.

The \lst"x-symbol" package may now work together with {\WhizzyTeX}, provided
you have a version of {\tt x-symbol} greater than v4.3.3.  However, the
option Page-to-point may still not work (under \lst"x-symbol", the cursor
would be inserted at the end of the buffer instead of at the
current-position).  Then, you may use the following alternative.

Alternatively, with (X)Emacs version 21 and above, you may set
\lst"whizzy-write-annotate" to nil to tell {\WhizzyTeX} not to use
\lst"write-region-annotation-functions". The slice will be write slower, but
in a way so that the \lst"x-symbol" package should always be compatible with 
{\WhizzyTeX}. 

You may also wish to set \lst"'x-symbol-auto-conversion-method" to
\lst"'fast".

\item How can a click in the \lst"advi" window reposition the point
in the emacs window?

First, you should be using the \lst"advi" previewer.  Then, check the
documentation of \lst"advi", since bindings may depend on versions. This is
likely to be \lst"shift-left-mouse" or \lst"left-mouse".

There might also be a short-cut that temporarily rebinds
\lst"shift-xxx-mouse" to \lst"xxx-mouse". This can be toggled by typing
\lst"e"   in the \lst"advi" or by calling \lst"advi" with the option
\lst"-edit". 
%
Note that \lst"shift-midddle-click" and \lst"shift-right-click" (or their
shortcut versions) are used for \lst"move" and \lst"resize" commands when
WhizzyEditting.


\end {enumerate}


\section {Using {\WhizzyTeX} with platex}

{\WhizzyTeX} has been designed to work with standard LaTeX.
However, it should also work  with other implementations of LaTeX, as long
as they allow the creation of new formats. 

As an example, to make it work with \lst"platex" you can edit 
the script \lst"whizzytex" and change some findings as follows
(in the experimental version): 
\begin{verbatim}
INITEX=iniptex
LATEX=platex
FORMAT=platex
\end{verbatim}

\hypertarget{cygwin}{}
\section {Using {\WhizzyTeX} under Windows}


{\WhizzyTeX} is designed for \lst"Unix" plateforms.  However, Marciano
Siniscalchi reported that it successfully worked on his
Window plateform under Cygwin (1.5.5-1), using \texttt{Cygwin/Xfree 4.3} and
\texttt{xdvi} from \texttt{tetex 2.0.2-13}.

The following modifications to the \texttt{whizzytex} script are
necessary\iffalse \space
(these have been contributed by Marciano Siniscalchi and Gregory Borota)\fi:
\begin{itemize}

\item 
Fix the \lst"xdvi" script by replacing \lst"xdvi.bin"
by \lst"exec xdvi.bin" on the last line of the script
(as in some Linux distribs).

\item
Fix the  \lst"whizzytex" script by replacing  the line
\begin{verbatim}
        preview() {mv $WHIZZY.dvi $WHIZZY.$VIEW; }
\end{verbatim}
with
\begin{verbatim}
        preview() {cp $WHIZZY.dvi $WHIZZY.$VIEW; rm $WHIZZY.dvi}
\end{verbatim}
(Permission to rename the .wdvi file is denied while
\lst"xdvi" is displaying it).

\item
Replace all occurrences of \texttt{ulimit} with \texttt{true} as
\texttt{ulimit} is not implemented on Cygwin. Then, when {\WhizzyTeX} falls
into a loop during latex-ing, which is possible on some input, {\WhizzyTeX}
will not be able to get it out. (When such cases occur, you'll have to quit
and restart whizzytex by hand---after your source file has been fixed.)

\item
Replace all instances of \texttt{wdvi} with \texttt{w.dvi}
(as \texttt{Yap} automatically  adds \texttt{.dvi} to a file name
not ending with \texttt{.dvi} and then complains about not finding the
file).

\end{itemize}
Unfortunately, Cygwin/XFree86 is quite slow compared to XFree86 under
Linux, etc. An alternative is to use a commercial X server, such as
XWin32 from StarNet (www.xwin32.com). MI/X from Microimages
(www.microimages.com) is another alternative, but since it is
XFree86-based, it is not as fast as XWin32.
Performance, even with XWin32, is still inferior to Linux.

\subsection*{Using \texttt{yap} instead of \texttt{xdvi} under Cygwin}

\def \yaptgz{whizzytex-yap-\YAPVERSION.tgz}
Gregory Borota has reported much better performances with 
\texttt{yap} than with \texttt{xdvi} under Cygwin.
He has written a small wrapper interface around \texttt{yap} to make it look
like \texttt{xdvi} for {\WhizzyTeX} and patched the source of WhizzyTeX
accordingly. See the README and INSTAL file coming with his
\href{\yaptgz}{\texttt{\yaptgz}} patch, 
\emph {coming with absolutely no warranty}. 

See also the detailed
\ahref {http://hellmund.dk/whizzy.html}{installation guide} written by
%\ahref {http://home.imf.au.dk/hellmund}
{Gunnar Hellmund}.

\section{Using Virtual Fonts with Active-DVI}

\href{http://cristal.inria.fr/advi/}{Active-DVI}
does not currently handle virtual fonts (when called with virtual fonts, it
usually shows a more or less blank screen). This can be fixed by expanding
virtual fonts with \texttt{dvicopy} prior to display as explained on the
{Active-DVI} FAQ.

In {\WhizzyTeX}, this can be done automatically by passing the option
\texttt{-dvicopy dvicopy} or equivalently setting
\begin{verbatim}
        DVICOPY=dvicopy
\end{verbatim}
in \texttt{.whizzytexrc} configuration file in the working directory.

\section{Known bogus latex packages}

\subsection{ieee.cls [2000/01/11]}

This package introduced a bug in the redefinition of \verb"\@xfloaf" 
that makes it incompatible with the \texttt{color} package

To fix it, your must redefine it as follows:
{\small
\begin{verbatim}
\def\@xfloat#1#2{\ifhmode \@bsphack\@floatpenalty -\@Mii\else
    \@floatpenalty-\@Miii\fi\def\@captype{#1}\ifinner
      \@parmoderr\@floatpenalty\z@
     \else\@next\@currbox\@freelist{\@tempcnta\csname ftype@#1\endcsname
       \multiply\@tempcnta\@xxxii\advance\@tempcnta\sixt@@n
       \@tfor \@tempa :=#2\do
                        {\if\@tempa h\advance\@tempcnta \@ne\fi
                         \if\@tempa t\advance\@tempcnta \tw@\fi
                         \if\@tempa b\advance\@tempcnta 4\relax\fi
                         \if\@tempa p\advance\@tempcnta 8\relax\fi
         }\global\count\@currbox\@tempcnta}\@fltovf\fi
    \global\setbox\@currbox \color@vbox \normalcolor \vbox\bgroup
    \def\baselinestretch{1}\small\normalsize
    \hsize\columnwidth \@parboxrestore}
\end{verbatim}}




\end{document}
