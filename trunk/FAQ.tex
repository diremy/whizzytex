\documentclass{article}

\usepackage {color}
\usepackage {hyperref}
\usepackage {hevea}

\title {\textcolor{blue}{WhizzyTeX}  Frequently Asked Questions}
\author {Didier R{\'{e}}my}
\date {Last Modified: \today}

\begin{document}

\maketitle
\let \lst \verb
\input {version}

\tableofcontents

\section {Where to find it}

URL is 
\href
{http://pauillac.inria.fr/whizzytex}{http://pauillac.inria.fr/whizzytex}

Last major release is 1.00

Last stable version is {\version}


\section {Known problems}

Fixed refers to last stable version.
\def \FIXED {\textit{Fixed in version > 1.00}}

\begin {enumerate}

\item
The \lst"x-symbol" package does not fully respect emacs
\lst"write-region-annotate-functions" protocol, but this is under fix.

With (X)Emacs version 21 and above, you may set
\lst"whizzy-write-annotate" to nil to tell WhizzyTeX not to use
the this feature. The slice will be write slower, but in a way
so that the \lst"x-symbol" package will be compatible with WhizzyTeX.

You may also wish to set \lst"'x-symbol-auto-conversion-method" to
\lst"'fast".

\item
Interactive commands \texttt{duplex} and \texttt{reformat} may not work in 
version~1.00. \FIXED.

\end {enumerate}


\section {Using WhizzyTeX with platex}

WhizzyTeX has been designed to work with standard LaTeX.
However, it should also work  with other implementations of LaTeX, as long
as they allow the creation of new formats. 

As an example, to make it work with \lst"platex" you can edit 
the script \lst"whizzytex" and change some findings as follows
(in the experimental version): 
\begin{verbatim}
INITEX=iniptex
LATEX=platex
LATEXFMT=platex
\end{verbatim}

\end{document}
