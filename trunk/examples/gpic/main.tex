\expandafter\ifx\csname SourceFile\endcsname\relax\else\SourceFile{main.ltx}\fi
\expandafter\ifx\csname Setlineno\endcsname\relax\def\SetLineno{0}\fi
%; whizzy -ext .tex -mkslice "../bin/mkgpic -slice" -mkfile "../bin/mkgpic -file"
%; whizzy -ext .tex -mkslice "make" -mkfile "make files"

\documentclass{article}

\title {WhizzyTeX with preprocessing}
\author {Didier R{\'e}my}
\let \lst \verb

\begin{document}

\maketitle

For illustration, we use a source file with inlined {\tt gpic} macros.
Such macros are surrounded by \lst".PS" and \lst".PE" markers on a single
line. The preprecessor is \lst"gpic -t" that transforms a file with such
macros into some latex code with \lst"\special" commands that more
DVI tools understand. 

WhizzyTeX must be told how to preprocess the slice and the file. 
A first solution would be to use the \lst"gpic -t" command for both.
However, one must first note that \lst"mkslice" command receives the target
file \lst"_whizzy_main.tex" as argument while the \lst"mkfile" command will
receive the target's name. So at least some analyses or source and target
must be done. Second, note gpic expansion does not preserve line numbers. 
This must then be correctly by inserting source line number information
before preprocessing the file. The macro \lst"\Setlineno" followed by a
number will tell WhizzyTeX to use that number for the current input line.
Since \lst"\inputlineno" is readable only, we instead compute the difference
and take that difference into account when printing line numbers. 

Since the preprocessor changes the name of the file, it should also insert
a command \lst"\SourceFile{main.ltx}" (or \lst"\SourceFile{subfile.ltx}")
at the beginning of the file. This is done by the script
\lst"../bin/mkgpic". 

Thus the following line:
\begin{quote}\small
\begin{verbatim}
%; whizzy -ext .tex -mkslice "../bin/mkgpic -slice" -mkfile "../bin/mkgpic -file"
\end{verbatim}
\end{quote}
None the specification of the extension \lst".tex", while the default value
is to take the extension of the master file, which would be \lst".ltx". 

We can also the command make. For simplicity, we let make call
\lst"mkgpic". 
\begin{quote}
\begin{verbatim}
SOURCES = main.ltx subfile.ltx
TEXFILES = $(patsubst %.ltx, %.tex, $(SOURCES))
GPIC=../bin/mkgpic

main.dvi: $(TEXFILES)
	latex main.tex

files: $(TEXFILES)

# to force recompilation even if same date (second is not precise enough)
.force:
_whizzy_%.tex: _whizzy_%.new .force
	$(GPIC) -slice $@

%.tex: %.ltx
	$(GPIC) -file $<

clean:
	rm -f *.{tex,log,aux,dvi}
\end{verbatim}
\end{quote}
All files should be processed before starting.

\subsection* {Other uses of preprocessing}

The use of {\tt gpic} macros with multiple files is already sophisticated. 
Preprocessing is simpler when either  line numbers are left unchanged, or 
when the document is composed of a single file. 

\subsection* {Examples of drawing}

Here is a drawing that need preprocessing:
\expandafter\ifx\csname graph\endcsname\relax \csname newbox\endcsname\graph\fi
\expandafter\ifx\csname graphtemp\endcsname\relax \csname newdimen\endcsname\graphtemp\fi
\setbox\graph=\vtop{\vskip 0pt\hbox{%
    \special{pn 8}%
    \special{pa 0 1500}%
    \special{pa 1350 1500}%
    \special{pa 1350 0}%
    \special{pa 0 0}%
    \special{pa 0 1500}%
    \special{da 0.050}%
    \graphtemp=\baselineskip\multiply\graphtemp by -1\divide\graphtemp by 2
    \advance\graphtemp by .5ex\advance\graphtemp by 0.000in
    \rlap{\kern 0.675in\lower\graphtemp\hbox to 0pt{\hss {Generic structures}\hss}}%
    \special{ar 650 350 400 150 0 6.28319}%
    \graphtemp=.5ex\advance\graphtemp by 0.350in
    \rlap{\kern 0.650in\lower\graphtemp\hbox to 0pt{\hss {Monoid}\hss}}%
    \special{ar 650 1000 375 150 0 6.28319}%
    \graphtemp=.5ex\advance\graphtemp by 1.000in
    \rlap{\kern 0.650in\lower\graphtemp\hbox to 0pt{\hss {Group}\hss}}%
    \special{pa 650 500}%
    \special{pa 650 850}%
    \special{fp}%
    \special{sh 1.000}%
    \special{pa 675 750}%
    \special{pa 650 850}%
    \special{pa 625 750}%
    \special{pa 675 750}%
    \special{fp}%
    \hbox{\vrule depth1.500in width0pt height 0pt}%
    \kern 1.350in
  }%
}%
\Setlineno=
98
$$
\box\graph
$$
End of the drawing.

Note: Dashed and dotted lines are not implemented yet in gpic.

\section {This is a subfile in a subdir}

Can you see it?

So we are Leaving it now.


\end{document}
