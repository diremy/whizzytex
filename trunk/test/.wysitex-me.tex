%; wysitex   paragraph -dvi "xdvi -s 5 -expert -xoffset 0 -sidemargin 0.6 -topmargin 1 -geometry 625x960+10+35" -pre make
%; wysitex   document -dvi "xdvi -s 5 -expert -geometry 625x960+10+35"
%; wysitex   section  -dvi "xdvi -s 5 -expert -xoffset 0 -sidemargin 0.6 -topmargin 1 -geometry 625x960+10+35"

% The previous three lines are possible configuration for wysitex. 
% They may not be there, then, the default will be used. 
% If not present the Time-stamp line will be added if necessary. 

\documentclass{article}

\let \lst \verb

\def \wysitex{\textbf {WysiTeX}}
\def \instruction #1{\par\medskip \noindent$\Rightarrow$ {\em #1}}

\begin{document}
\pagestyle {empty}

\author {Didier R{\'e}my}

\title {
{\huge \wysitex}
\\[1em]
{\em An {\bfseries Emacs mode} 
for incremental display of \\ 
{\bfseries {\LaTeX} documents}}
}

\maketitle 

\begin{abstract}
{\wysitex}\footnote {standing for {\em {\bf W}hat {\bf Y}ou {\bf S}ee
{\bf I}s what you {\bf TeX}}} is an emacs minor mode for incrementally
TeXing and Viewing the file while editing.
%
It is pre-configured to work with ghostview-based and xdvi-based previewers.

{This documentation is also a test example. That is, you can test it yourself
by visiting this source file in {\wysitex} mode (and modify it as you
wish). }
\instruction 
{Please move the cursor foward}
\end{abstract}


\section {Principle}%% 1.0.0.

In short {\sc \wysitex} is displaying a small slice of the document around
the cursor. 

This is a two-part engine: 

\instruction 
{Move the cursor to the subsection you wish to display/edit}

\subsection {Emacs is watching you} typing and moving in the buffer attach%% 1.1.0.
to the source file in keep saving the region (slide, section, or subsection)
your are in  into a spooler file. 

\subsection {A shell-script is watching the spooler file}%% 1.1.1.
and keep recompiling it as soon as it is refreshed.


\section {Test} %% 1.1.2.

\instruction
{You can use this section to test in editing mode: just type in some 
latex code.}

This is a test empty enumeration: 
\begin {enumerate}

\item

\item

\end {enumerate}

\instruction
{You may now continue reading this document and come back to this
section for more tests later.}

\paragraph{Remark:}
A more interesting test file is \lst"wysitex-slides.tex". 


\section {Implementation}%% 2.2.0.

The section briefly describe the implementations, composed of three parts. 

\subsection {Hacks}%% 2.3.0.

\subsubsection* {Emacs hacks}

Besides administrative business, the main trick is to use 
\lst"post-command-hook" to make emacs watching if anything has changed. 

It uses \lst"buffer-modified-tick" to tell if any editing has actually
occurred, and compare the point position with the (remembered) position of
the region being displayed to see if saving should occur. 

Last, it uses \lst"sit-for" to delay savings until idleness or a
significant number of editing changes. 

\subsubsection* {TeX hacks}

The only TeX hack to build a format file so as to avoid reloading the whole
macros at each compilation. This is entirely transparent, that is, the
source file does not have to understand this.

The hack is to redefine \lst"\documentclass" which in turn  redefines
\lst"\document" to execute \lst"\dump" (after redefining \lst"\document"
to its old value and \lst"\documentclass" so that it skips everything till
\lst"\document". (This is rather robust, as it works with my preferred macros
latex package \lst"localmacros".)

The formatted file also includes a command \lst"\WysitexInput" that can be
used to dump page and section numbers while it is processed in an auxilliary
file. This is used by emacs to annotate the source file with these numbers
as latex comments. 

\subsubsection* {Unix hacks}

Mainly, a unix shell-script is watching the spooler file.  It recompiles the
format file (and the page and section number, but in batch mode) whenever
the source file (its unix date) has changed  and 
recompiles the slice whenever it is present (since wysitex renames ---hence
removes--- the slice before processing it).

If the file has been recompiled successfully, it triggers the previewer
(ghostscript or xdvi) so that it rereads the dvi or ps file. Otherwises, it
cat the log (hence into the \lst"*TeX-shell*" buffer where it has been
launched). 


\subsection {Interaction between the components}%% 3.3.1.

The control is normally done by emacs, which launches and kill the unix
deamon. However, for robustness, the unix deamon will commit suicide if the
previewer is dead, and in turn, the {\wysitex} mode will automatically turn
itself off when it notices that deamon has dyied. This allows to turn off
the {\wysitex} mode, simply by exiting the previewer. 

To that purpose, the deamon pid is saved in a file. 
This is also used to prevent several deamon running on the same file. 
Then {\wysitex} must be called with \lst"-kill" option to cleanup an  old
running session, which is done automatically when the mode is turn off.


\section {Short manual}%% 3.3.2.

This section is a very short manual that describes how to install, use and
parameterize WysiTeX. 

\subsection {Installation}%% 3.4.0.

The file \lst"wysitex.el" should be installed in the appropriated directory, 
the command \lst"wysitex-mode" can be made autoload (and, for instance, can
bound to the keep \lst"^C-^W", in latex-mode-map). 

The file \lst"wysitex" should be make executable, and visible in the emacs
executable path (or the variable \lst"wysitex-command-name" should be
adjusted in the file \lst"wysitex.el"). 

The file \lst"wysitex.sty" should be installed at a place so that it is
visible  from initex, or the DUMP variable should be adjusted in
\lst"wysitex".  

The shell-script \lst"wysitex" is written in bash and works for linux 6.2. 
It should work with other systems as well, but some adjustements may be
needed. 


\subsection {Configuration}%% 3.4.1.

All parameters are assigned default values, so WysiTeX should do something
reasonable on any latex file.
Configuration can be done by setting emacs variables (for instance via a
wysitex-mode-hook) or by setting file-dependent parameters by 
inserting configurations lines among the first lines of the file. 


\paragraph {Configuration lines}

A configuration line is of the form:
$$
\lst`"%; *\\(wysitex[^ \n]*\\) +\\([^\n]\\) *$"`
$$
The first group is one of the two keywords \lst"wysitex" and 
\lst"wysitex-paragraph" recognized so far; the second group are arguments. 
If two confirgurations lines appear with the same keyword, only the first
one is taken into account. 
\begin{description}
\item[wysitex]
The is the main configuration command. The arguments should match the
regular expression: 
$$
\lst`"\\([^ \n]+\\) +\\([^ \n]+\\) +\\([^\n]*[^ \n]\\) *$"`
$$
They determine in order the mode, the previewer type, followed by the
previewer command and other optional aguments to be passed to the wysitex
program.  A the right most groups are optional.  The mode controls how the
document is sliced. The previewer tells wether to use dvi-based or ps-based
previewers and defines the command to call the previewer (the name of the
file will be added to the command strings at the end).

\item[wysitex-paragraph]
This sets the emacs-variable wysitex-paragraph to the \lst"<args>"
which should be a regular-expression string. 

\end{description}
The first group defines the mode to be used, as described in the next
subsection. The second group the previewer type and its value should be either
\lst"-dvi" or \lst"-ps". 
The third group, that is, the remaining of the configuration line as 
as list of arguments that will be passed to the \lst"wysitex" shell
script. Arguments that contain blanks should be quoted.

The first argument, which is mandatory, defines the previewer command that
will be used to call the previewer (the name of the file will be appended).

The second and third optional arguments if \lst"-pre <command>" defines
the preprocessor used to build the tex file from the slice file. 
The command will be passed the name of the target file
\lst"_wysitex_$BASICFILENAME.tex", which it should produce from the source 
file \lst"_wysitex_$BASICFILENAME.new" (the under which wysitex renames 
the slice saved by emacs). 
Here \lst"$BASICFILENAME" is the name of the file under \lst"WysiTeX"
stripped off its suffix. By default, \lst"wysitex" will the source to
the target. 

\paragraph {Cool bindings}
The emacs mode defines the command \lst"wysitex-next-slice" and
\lst"wysitex-previous-slice", to move from slices forward or backward.
Binding these to appropriate keys will help you move pages from the emacs
buffer. 


\subsection {Modes}%% 4.4.2.

WysiTeX recognizes three modes \lst"slide", \lst"section", and \lst"document". 
The mode determines the slice of the document beeing displayed and how
frequently updates occurs. 
\begin{description}

\item [slide]

The mode \lst"slide"  is mainly used for documents of the class seminar. 
In slide mode, the slide is the text bewteen two \lst"\begin {slide}"
comments (thus,  the text between two slides is displayed after the
preceeding slide).  

\item [section]
In \lst"section" mode, the slice of text is the current chapter, section, or
subsection.

\item [paragraph]
The \lst"paragraph" mode is a variation on section mode where, the separator
\lst"wysitex-paragraph" is defined by the user (set to two empty lines by
default) instead of using \lst"\section"  and \lst"subsection" commands. 
subsection.

\item [document]
In \lst"document" mode, there is no sectionning unit and the whole document is
recompiled altogether.

\end{description}
Both \lst"section" and \lst"paragraph" modes sets on the ``marks'' mode
that tells the shell to recompilation section and page numbers in batchmode
so that emacs can annotation sections commands with this numbers, and
finally tell tex to adjust the numbers accordingly.

There are several possible (future?) improvements to the \lst"section" mode:
\begin {enumerate}

\item
An \lst"auto" mode could decide  at subsections or sections only could be made
depending on the (source code) length of the current section. 

\item
The enclosing section title could be included when in a subsection.
(Actually, all sections could be displayed, moving into sections and
subsections acting as opening drawers.

\end {enumerate}


\subsection {Using the command make}%% 5.4.3. 

This is made possible by using the \lst"-pre" option on the command line
to  define a command that will be used to process the slice 
\lst"_wysitex_$BASICFILENAME.slc" saved by emacs
into the file \lst"_wysitex_$BASICFILENAME.slc" to be be processed 
by latex.

The command command \lst"make" can be used as a preprocessor (with an
appropirate \lst"Makefile").  However, one may have to work around
\lst"make"'s notion of time, which is usually too rought. 
This is safe, since
\lst"WysiTeX" tests itself for needed recomputations (it includes
a slice-version stamp on the first line of the slice file for that
purpose). 

\subsection {Frequency of recompilation}%% 5.4.4.

To obtain maximum {\wysitex} effect, a new slice should be save after any
edition changed or any displacement that outside of the current slice.
However, to avoid overloading the machine or useless and anoying
refreshments, some compromise can is made: when either edition or movement
is continuous, saving a new slice is delayed until at least 10 editing
changes. The continuity is controlled in miliseconds by the emacs variable
\lst"wysitex-pause". This variable is set to \lst"200" in slide
mode, to \lst"1000" in section mode, and to \lst"2000" in document mode.

The format is automatically recompiled at the beginning of each session, and
whenever the buffer continaing the file is saved. That is, to load use new
packages or define new global macros (before the \lst"\begin{document}"), it
sufficies to save the current file.

\subsection {Errors}%% 6.4.5.

When latexing a slice produces an empty file or an error, the display is
unchanged. Instead, the log of the session is printing in the emacs
tex-shell buffer. 

If an error occur when building the initial format, the log of the session
is also displayed, but then, the mode is turned off. That is, a session can
only be started in a configuration where the format can be correctly build. 

If a reformating error occurs, the display won't be changed until the format
can correctly be rebuilt.

\end{document}
