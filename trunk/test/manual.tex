%; whizzy   document -dvi "dview -whizzy"
%; whizzy   document -dvi "xdvi" -pre "make -f Makefile.tpl"
%; whizzy   paragraph -dvi "xdvi" 
%; whizzy   section  -dvi "xdvi"

% The previous three lines are possible configuration for whizzytex. 
% They may not be there, then, the default will be used. 
% If not present the Time-stamp line will be added if necessary. 

\documentclass{article}

\topmargin 0em
\headsep 0in
\headheight 0in
\evensidemargin 0em
\oddsidemargin 0em

\usepackage {color}
\let \lst \verb

\def \whizzy{{Whizzy\kern -0.3ex\raise 0.2ex \hbox{\TeX}}}
\def \Whizzy{\textbf {\textcolor {blue}{\whizzy}}}
\def \instruction #1{\par\medskip \noindent$\Rightarrow$ {\em #1}}

%% The following lines are to help HeVeA make  the HTML version of the manual

%HEVEA \def \instruction #1{}
%HEVEA\def \whizzy{{Whizzy{\TeX}}}
%HEVEA\renewcommand{\@bodyargs}{TEXT=black BGCOLOR=white}

\begin{document}
\pagestyle {empty}

\author {Didier R{\'e}my}

\title {
{\huge \Whizzy}
\\[1em]
{\em An {\bfseries Emacs mode} 
for incremental display of \\ 
{\bfseries {\LaTeX} documents}}
}

\maketitle   


\begin{abstract}
\def \B{\textbf}
{\whizzy}\footnote {Standing for {\em {\B W}hat {\B i} {\B z}ee
{\B i}z what {\B y}ou {\expandafter \B \TeX}}} is an emacs minor mode for
incrementally ({\TeX}ing and) previewing the file while you are editing.
%
It is pre-configured to work with ghostview-based and xdvi-based previewers.
This emacs-mode comes with two companion files ---a {\LaTeX} package and a
shell script acting as deamon. 

\instruction 
{This manual can be viewed from emacs in {\whizzy} {\tt paragraph}
mode. Then, it is self-illustrating and you may follow these instructions;
Otherwise, please ignore them.}

\instruction {Move the cursor foward in the editor to continue.}
\end{abstract}


\section {Principle} 

In short, {\sc \whizzy} is selecting a small slice of the document that 
you are editing around the cursor (according to the selected mode) 
and redisplay the slice incrementally as it changes through edition. 

\paragraph {Emacs is watching you} typing and moving in the 
emacs buffer attached to the {\TeX} source file that your editing and keeps
saving the current slice (current slide, section, or subsection, according
to the mode) into {\bf a pool}.

\paragraph {A shell-script is watching the pool}
and keeps recompiling when some new slice (or other companion files) 
are made available in the pool, and, whenever recompilation succeeds, it 
updates the display of the slice. 

\instruction 
{Move the cursor to different parts of the documments, to observe 
slices both in emacs and on the display. }

\instruction {Move to the next section for an interactive demo}


\section {Test}

\instruction
{You can use this section to test in editing mode.
You may want to save a copy of the file first :-). 
Just type in some latex code below.}

This is an empty enumeration that you can fill in: 
\begin {enumerate}
\item
\item
\end {enumerate}

\instruction
{You may now continue reading this document and come back to this
section for more tests later.}

\paragraph{Remark:}
A more interesting test file is \lst"slides.tex".

\paragraph{Errors:}
Note that, while you are editing the document, it may temporarily be
ill-formed. That is, \TeX-ing the slice will fail. In this case, the slice in
not redisplayed; instead, {\whizzy} tries to hilight the ill-formed region.
(In fact, the region where \TeX\ detects some inconsistency, which sometimes
may be much further that the error itself). The hilighting disappears as
soon as the slice is successfully \TeX-ed (and re-displayed). 

When an error occurs the {\whizzy} shell window also pops up after a few
seconds of latency (if you keep typing, {\whizzy} asumes that you are in the
middle of editing some {\TeX} macro or environment only highlights the
error). 

\instruction
{You may intentionally typed some incorrect tex to see how the error is
highlighted.}


\section {Implementation}

This section briefly describe the implementations, composed of three parts.

\subsection {Code} 

\subsubsection* {Emacs code}

Besides administrative business, the main trick is to use 
\lst"post-command-hook" to make emacs watch changes. 
It uses \lst"buffer-modified-tick" to tell if any editing has actually
occurred, and compare the point position with the (remembered) position of
the region being displayed to see if saving should occur. It also uses
\lst"sit-for" to delay savings until idleness or a 
significant number of editing changes. 

{\whizzy} can also show the current point on the display, in which case
slices are also recomputed when the cursor moves, but with lower priority.

\subsubsection* {TeX code}

The main TeX hack is to build a format file so as to avoid reloading the
whole macros at each compilation. This is (almost\footnote{{\tt
$\backslash$begin\{document\}} should be typed as such without any white
white space}) entirely transparent, that is, the source file does not have
to understand this.

The hack is to redefine \lst"\documentclass" which in turn  redefines
\lst"\document" to execute \lst"\dump" (after redefining \lst"\document"
to its old value and \lst"\documentclass" so that it skips everything till
\lst"\document"). This is rather robust ---it even works with 
complex macros such as my prefered package \lst"localmacros" that reads
macros at the end of the document.

There are a few other less important hacks. For instance, the following
macros are also made available (but only for expert use):
\begin {itemize}
\item
\lst"\WysitexInput" is used by the shell-script to tell {\TeX} to dump page
and section  numbers in some format that is later processed by unix to be
passed back to emacs as eamcs-lisp code.
\item
\lst"\WysitexTex" is used by emacs to tell {\TeX} to adjust counters to
the current section. 
\item
\lst"\WhizzyLine" is used by emacs to tell where the cursor is.
Then \TeX\ will diplsay the cursor if possible, {\em ie.} if the cursor is
not in a special mode. This usually works fine, but may sometimes interfere
with the output. 
\end {itemize}
When compiling the format, {\whizzy} also look for a file of name
\lst"whizzy.sty", which if existing is loaded at the end of the macros. 
This may be used to add other macros in {whizzy} mode, {\em eg.} 
some {\TeX} environments may be redefined to changed they type setting,
according to whether the current line is inside or outside the environment. 
(We have written such an extension for an exercise package that sends the
answers at the end in an appendix, unless the cursor is inside the answer,
in which case the answer is inline.)

\subsubsection* {Unix code}

Mainly, a unix shell-script is watching the spooler file.  It recompiles the
format file (and the page and section number, but in batch mode) whenever
the source file (its unix date) has changed  and 
recompiles the slice whenever it is present (since {\whizzy} renames ---hence
removes--- the slice before processing it).

If the file has been recompiled successfully, it triggers the previewer
(ghostscript or xdvi) so that it rereads the dvi or ps file. Otherwises, it
process the {\TeX} log file and tries to locate the error. It then sends part
of the log file with annotations to the \lst"*TeX-shell*" buffer from which
emacs has been {\whizzy}, so that emacs can repport the error. 


\subsection{Some dirty hacks}

{\whizzy} works best with the \lst"advi" previewer, as explained below. 
However, it also works with \lst"xdvi" and \lst"gv". 

Sending the signal \lst"-HUP" to \lst"gv" should make it reopen the file.
However, \lst"gv" does not execute the update if the newer version has the
same time stamp as the older version. Fortunately, it executes the reopen if
the new version has a modified time older than the old version :-) Here the
ouput postcript file  is antidated whenever necessary to force \lst"gv"
to reopen it.


\subsection {Interaction between the components} 

The control is normally done by emacs, which launches and kill the unix
deamon. However, for robustness, the unix deamon will commit suicide if the
previewer is dead, and in turn, the {\whizzy} mode will automatically turn
itself off when it notices that deamon has dyied. This allows to turn off
the {\whizzy} mode, simply by exiting the previewer.

To that purpose, the deamon pid is saved in a file. 
This is also used to prevent several deamon running on the same file. 
Then {\whizzy} must be called with \lst"-kill" option to cleanup an  old
running session, which is done automatically when the mode is turn off.


\section {Short manual} 
This section is a very short manual that describes how to install, use and
parameterize {\whizzy}. 

\subsection {Installation}

The file \lst"whizzytex.el" should be installed in the appropriated directory, 
the command \lst"whizzytex-mode" can be made autoload (and, for instance, can
bound to the keep \lst"^C-^W", in latex-mode-map). 

The file \lst"whizzytex" should be make executable, and visible in the emacs
executable path (or the variable \lst"whizzy-command-name" should be
adjusted in the file \lst"whizzytex.el"). 

The file \lst"whizzytex.sty" should be installed at a place so that it is
visible  from initex, or the DUMP variable should be adjusted in
\lst"whizzytex".  

The shell-script \lst"whizzytex" is written in bash and works for linux 6.2. 
It should work with other systems as well, but some adjustements may be
needed.

\subsection {Configuration} 

All parameters are assigned default values, so WysiTeX should do something
reasonable on any latex file.
Configuration can be done by setting emacs variables (for instance via a
whizzytex-mode-hook) or by setting file-dependent parameters by 
inserting configurations lines among the first lines of the file.

Turning the {\whizzy} mode on, may ask the user to confirm or changed the
default values if no local value has be specified (or if required explicity
by appropriate optional argument).


\subsubsection {Calling {\whizzy}}

The main command is \lst"whizzytex-mode" is used to activate or desactive 
{\whizzy} mode. It takes one optional parameter. 

Without arguments it switches on and off the whizzytex-mode. With a zero or
negative integer as argument it switches the mode off. With a strictly
positive interger argument, it attempts to turn the mode on. This will
examine the file for local configuration parameters, or for guessing a best
mode for the file. If need be it will ask the user to confirm the default
choises.

If the argument is present and with a numeric value equal to 4, it will
always ask the user to confirm (and let him change) the configuration
variables.

Emacs also offers several optional commands to move from slice to slice
(see \lst"whizzy-suggested-hook" in the \lst"whizzytex.el" file). 
There are also experimental commands for turning pages in the displayed
document. 


\subsubsection {File-based configuration}

A configuration line is matched by the regular expression:
$$
\lst`"^%; *\\(whizzy[^ \n]*\\) +\\([^\n]\\) *$"`
$$
and should appear in the first 400 bytes of the file. 

The first group is a configuration keyword. 
So that the only-two  keywords are \lst"whizzy" and 
\lst"whizzy-paragraph". If two confirgurations
lines have the same keyword, only the first one is taken into
account.
\begin{description}
\item[whizzy]
This is the main configuration line and should match the following regular
expressions (otherwise, {\whizzy} will assigned default values to the
missing arguments): 
$$
\lst`"\\([^ \n]+\\) +\\([^ \n]+\\) +\\([^\n]*[^ \n]\\) *$"`
$$
The first group determines the mode, {\i.e. em} the type of document, and
accordingly how it should be sliced.  The different modes are descriped in
section~\ref {modes}.

The second group defines the previewer type and are described in
section~\ref{types}.  The third group, called is the previewer parameter
whose semantics depends on the previwer type.

\item[whizzy-paragraph]
This sets the emacs-variable whizzy-paragraph to the \lst"<args>"
which should be a regular-expression string. 
\end{description} 


% \paragraph {File names} 
% The command will be passed the name of the target file
% \lst"_whizzy_$BASICFILENAME.tex", which it should produce from the source 
% file \lst"_whizzy_$BASICFILENAME.new" (the under which whizzy renames 
% the slice saved by emacs). 
% Here \lst"$BASICFILENAME" is the name of the file under \lst"WysiTeX"
% stripped off its suffix. By default, \lst"whizzy" will the source to
% the target. 
%$


\subsubsection {Emacs global configuration}

\paragraph {Emacs cool bindings}

The emacs mode defines the command \lst"whizzy-next-slice" and
\lst"whizzy-previous-slice", to move from slices forward or backward.
Binding these to appropriate keys will help you move pages from the emacs
buffer. 

These can be define using \lst"whizzytex-mode-hook". 
{\whizzy} does not do any default binding for you, since those are often
annoying when in conflict with personal bindins. 
However, it defines a function \lst"whizzy-suggested-hook", so that you can
get default bindinds by adding the line
\begin{verbatim}
    (add-hook whizzytex-mode-hook 
        (function whizzy-suggested-hook))
\end{verbatim}
to your \lst".emacs" file. 
Then, you will get most useful binding (see only help). 
In particular, this will install a \lst"Whizzy" menu, from which you can
call for help.

Similarly, you can remove the other optional features, 
such as error-display that are on by default, by adding the line
\begin{verbatim}
    (add-hook whizzytex-mode-hook 
        (function whizzy-remove-options-hook)
\end{verbatim}
to your \lst".emacs" file.

We also recommend the line:
\begin{verbatim}
    (autoload 'whizzytex-mode "whizzytex" 
       "Local mode for interactive previewing of TeX" t)
\end{verbatim}

\paragraph {Emacs configuration variables}

You can also change other configuration parameters globally by changing the
default values of whizzytex-mode emacs variables. 
Configurable variables are all listed at the beginning of the
\lst"whizzytex.el" file. You can change them directly, or using a 
the \lst"whizzytex-mode-hook". 


\subsection {Modes} 
\label {modes}

WysiTeX recognizes three modes \lst"slide", \lst"section", and \lst"document". 
The mode determines the slice of the document beeing displayed and how
frequently updates occurs. 
\begin{description}

\item [slide]

The mode \lst"slide"  is mainly used for documents of the class seminar. 
In slide mode, the slide is the text bewteen two \lst"\begin {slide}"
comments (thus,  the text between two slides is displayed after the
preceeding slide).  

In slice modes, overlays are ignored {\em i.e.} all overlays all displyed in
the same slide, unless a command
\lst"\overlay {"$n$\lst"}" occurs on the left of the point, on the same line
(if several commands are on the same line, the 
right-most one is taken), in which case only layers $p \le n$ are displayed.

\item [section]
In \lst"section" mode, the slice of text is the current chapter, section.

\item [subsection]
As \lst"section" but also slice at subsections. 

\item [paragraph]
The \lst"paragraph" mode is a variation on section mode where, the separator
\lst"whizzy-paragraph" is defined by the user (set to two empty lines by
default) instead of using \lst"\section"  and \lst"\subsection" commands. 
subsection.

\item [document]
In \lst"document" mode, there is no sectionning unit and the whole document is
recompiled altogether. Currently, pages are not turned to point and the 
cursor is not shown in \lst"document" mode, because full document is not
sliced. (A slicing document mode could be obtained by working in paragraph
mode, with an appropriate regexp.)


\end{description}
Both \lst"section" and \lst"paragraph" modes sets on the ``marks'' mode
that tells the shell to recompilation section and page numbers in batchmode
so that emacs can annotation sections commands with this numbers, and
finally tell tex to adjust the numbers accordingly.

There are several possible (future?) improvements to the \lst"section" mode:
\begin {enumerate}

\item
An \lst"auto" mode could decide  at subsections or sections only could be made
depending on the (source code) length of the current section. 

\item
The enclosing section title could be included when in a subsection.
(Actually, all sections could be displayed, moving into sections and
subsections acting as opening drawers.

\end {enumerate}

\subsection {Previewer types}
\label {types}

The previewer types can have three possible values: 
\begin{description}
\item[{\bf {\tt -dvi} or {\tt -ps}}]\indent

These indicate that file should be processed down to \lst"dvi" or
\lst"Postscript" format respectively. Then, the previewer parameter is the
command to call 
the previewer.  This string will be passed as such to to the {\whizzy}
shell-script. Note that the name of the \lst"dvi" or postscript file will be
appended to the previewer's command.  Actually, other optional parameters to
the {\whizzy} shell-script can be added at the end of the previwer
parameter.  Arguments that contain blanks should be quoted (see \ref
{make}).  (For instance \lst"-pre <command>" can be used to define a
preprocessor to transform the slice into an input file for latex (\lst"cat"
is used by default).

\item[{\bf {\tt -master}}]\indent

This previewer type must be used when the file is not a {\LaTeX} document
but only a {\LaTeX} file included in a {\LaTeX} document. 
In this case, the third group is the name of the master file of the {\LaTeX}
document, {\em i.e.}, the one on which the \lst"latex" command can be run.

When a file is masterred, turning {\whizzy} mode on requires that another
buffer is already visiting the master file and that this buffer is in
{\whizzy} mode. Turning {\whizzy} mode off in a mastered file only affects
the local buffer and leaves {\whizzy} running on the master file. 

When the file is mastered, the document mode will only slice the current
file, not the entire {\LaTeX} document of the master file.

\end{description}


\subsection {Using the command make} 
\label {make}

This is made possible by using the \lst"-pre" option on the command line
to  define a command that will be used to process the slice 
\lst"_whizzy_$BASICFILENAME.slc" saved by emacs
into the file \lst"_whizzy_$BASICFILENAME.slc" to be be processed 
by latex.

The command command \lst"make" can be used as a preprocessor (with an
appropirate \lst"Makefile").  However, one may have to work around
\lst"make"'s notion of time, which is usually too rought. 
This is safe, since
{\whizzy} tests itself for needed recomputations: it does so by dropping any
new version in a pool directory (\lst"_whizzy_$BASICFILENAME/"). 

\subsection {Watching other files}

{\whizzy} is designed to watch other files than just the slice saved by
emacs. In fact, it watches any file dropped in the pool directory. 
For instance, 
if your source file uses images, you can just change the image and
drop the new version in the pool. Then {\whizzy} will pick the new version,
move it to the working directory and recompile a new slice. Be aware of name
clashes: if you drop a file in the pool, it will automatically be move to
the working directory with the same name, overriding any file of the same
name sitting there. 

By default, {\whizzy} send itself a \lst"kill -STOP" signal when it finds
nothing in the pool (this allows you to let {\whizzy} inactive for hours
without consuming any CPU. Correspondingly, when emacs drops a new file in
the pool, it sends a \lst"kill -CONT" signal. If you want to watch other
files and not have to deal with waking up {\whizzy} you can pass the extra
option \lst"-nonstop" to the \lst"whizzytex" command-line; then it will sleep
instead of stopping itself, and will periodically watch for news files 
in the pool without you need to signal it to continue. 
In case, you want to keep with the stop-mode and wake it up explicitly, the
process number is stored in the file \lst"._whizzy_$BASICNAME.tex".


\subsection {Frequency of recompilation} 

To obtain maximum {\whizzy} effect, a new slice should be save after any
edition changed or any displacement that outside of the current slice.
However, to avoid overloading the machine or useless and anoying
refreshments, some compromise can is made: when either edition or movement
is continuous, saving a new slice is delayed until at least 10 editing
changes. The continuity is controlled in miliseconds by the emacs variable
\lst"whizzy-pause". This variable is set to \lst"200" in slide
mode, to \lst"1000" in section mode, and to \lst"2000" in document mode.

The format is automatically recompiled at the beginning of each session, and
whenever the buffer continaing the file is saved. That is, to load use new
packages or define new global macros (before the \lst"\begin{document}"), it
sufficies to save the current file.

\subsection {Cross-references, page and section numbers} 

The slice is always recompiled with the \lst".aux" file of the whole
document.  In paragraph mode, cross references and section numbers are 
recompiled whenever the buffer itself is saved (manually). 
This recompilation operates in batch and concurrently with the recompilation
of slices, so it may take several slices before the new counters or
references are adjusted.

(The recompilation of the whole document is off in slide mode.) 


\subsection {Errors} 

When  latexing a slice produces an empty file or an error, the display is
unchanged. Instead, {\whizzy} attempts to localize the error in the editing
bufffer. First it computes the word difference with the previous sucessfully
recompiled slice. If this succeeds and the difference region is visible
(appended or changed) and connexe, then this difference  is highlighted;
otherwise, the log of the session is printing in the emacs tex-shell buffer, 
and an attempt is made to localize and highlight the text in the editing
buffer that is involved with the error.


If an error occur when building the initial format, the log of the session
is also displayed, but then, the mode is turned off. That is, a session can
only be started in a configuration where the format can be correctly build. 

If a reformating error occurs, the display won't be changed until the format
can correctly be rebuilt.

\end{document}

\end{document}
