% manual.tex --- WhizzyTeX, a wysiwyg environment for TeX
%
%  Copyright (C) 2001, 2002, 2003, 2004, 2010, 2011, 2013, 2020 INRIA
%
%  Author         : Didier Remy <Didier.Remy@inria.fr>
%  Version        : 1.3.6
%  Bug Reports    : whizzytex-bugs@inria.fr
%  Web Site       : http://gallium.inria.fr/whizzytex
% 
%  WhizzyTeX is free software; you can redistribute it and/or modify
%  it under the terms of the GNU General Public License as published by
%  the Free Software Foundation; either version 2, or (at your option)
%  any later version.
%  
%  WhizzyTeX is distributed in the hope that it will be useful,
%  but WITHOUT ANY WARRANTY; without even the implied warranty of
%  MERCHANTABILITY or FITNESS FOR A PARTICULAR PURPOSE.  See the
%  GNU General Public License for more details 
%  (enclosed in the file GPL).
%  
%  See the file COPYING enclosed with the distribution.
%%%%%%%%%%%%%%%%%%%%%%%%%%%%%%%%%%%%%%%%%%%%%%%%%%%%%%%%%%%%%%%%%%%%%%%%%%%%
%  File manual.tex (WhizzyTeX documentation source file)
%%%%%%%%%%%%%%%%%%%%%%%%%%%%%%%%%%%%%%%%%%%%%%%%%%%%%%%%%%%%%%%%%%%%%%%%%%%%

\documentclass[12pt]{article}

\input {version}
\usepackage {manual}

\begin{document}
\begin{rawhtml}
<link rel="Top" href="index.html">
<link rel="Up" href="#htoc">
<link rel="Contents" href="#htoc">
<link rel="Section" title="Installation" href="#install">
<link rel="Section" title="Using WhizzyTeX" href="#using">
<link rel="Section" title="Manual" href="#manual">
<link rel="Section" title="Viewers" href="#viewers">
<link rel="Section" title="Customizing" href="#custom">
<link rel="Section" title="WhizzyEditing" href="#edit">
<link rel="Section" title="Implementation" href="#impl">
\end{rawhtml}
\pagestyle {empty}
\author {\rightline {Didier R{\'e}my}}
\date {\hfill Version {\version}}
\begingroup
\let \@\relax
\title {
\hfilneg{\huge \Whizzy
\footnote{{\whizzy} is free software, 
Copyright \copyright 2001, 2002, 2003, 2004, 2010, 2011, 2013, 2020 INRIA
and distributed under the GNU General Public License
(See the COPYING file enclosed with the distribution).}}
\\[1em]
{\em
\ifhevea
An {\bfseries Emacs} minor-mode \\
for {\bfseries incremental viewing of} \\ 
{\bfseries {\LaTeX} documents}
\else
\vbox {%
\centerline
{{\raggedright \normalsize \activedvi {Whizzy-}{Te\/X}{WhizzyTeX}}
\hfill
{\raggedright \normalsize \emacs {Whizzy-}{Te\/X}{WhizzyTeX}}
\hfill
\vtop {\vskip -5em \hbox {%
\begin{tabular}{r@{}}
An {\bfseries Emacs} minor-mode for\\
{\bfseries incremental viewing of} \\ 
{\bfseries {\LaTeX} documents}
\end{tabular}}}}\vskip -1em}
\fi
}
}

\maketitle   
\endgroup

\begin{abstract}
\def \B{\textbf}
{\Whizzy}
% \footnote {Standing for {\em {\B W}hat {\B i} {\B z}ee {\B i}z what
% {\B y}ou {\expandafter \B \TeX}}}
is an Emacs minor mode for incrementally
viewing {\LaTeX} documents that you are editing.
%
It works under Unix with {\tt gv} and {\tt xdvi} viewers, but 
the \href{http://gallium.inria.fr/advi/}{\ActiveDVI} viewer will
provide much better visual effects and offer more functionalities.

In addition, when used with {\ActiveDVI}, {\Whizzy} allows for mouse
edition of dimensions and floats, which can be used to adjust spaces,
move or resize objects visually. 
\end{abstract}

\newpage
\tableofcontents


%HEVEA\hypertarget{install}{}

\section {Installation}

\subsection {Requirements}

{\whizzy} is designed for \lst"Unix" plateforms\footnote{It has 
been reported to successfully work on Windows under Cygwin---See the 
\href{FAQ.html}{FAQ}.}.

To use {\whizzy}, you need {\tt Emacs} or {\tt XEmacs}, some standard 
{\tt latex} distribution, 
{\tt bash}, and at least one DVI, Postscript or PDF previewer, such as
{\tt advi}, {\tt xdvi}, or {\tt dvips} combined with {\tt gv}, or 
\texttt {xpdf}.

{\whizzy} has been developed under Linux but has not been extensively tested
on other platforms. However, {\LaTeX} and Emacs are quite portable and
possible compatibility problem with the bash shell-script should be minor
and easily fixable. Hence {\whizzy} should work with all distributions of
\texttt{latex} that are compliant to the standard. 

\subsection {Get the source}
Get the source {\tt whizzytex-\version.tgz} 
from the \href{http://gallium.inria.fr/whizzytex}{distribution}, 
uncompress and untar it in some working directory, as follows:
\begin{quote}
\begin{tt}
gunzip whizzytex-\version.tgz\\
tar -xvf whizztex-\version.tar\\
cd whizzytex-\version
\end{tt}
\end{quote}
Then, the installation can be automatic (default or customized), or manual.


\subsection{Warning!}
Many Linux installations make \lst"xdvi" a shell-script that erroneously
end with the line \lst#xdvi.bin "$@"# instead of 
\lst#exec xdvi.bin "$@"#. The later is needed to preserve the process 
id, so that signals sent to \lst"xdvi" are correctly received and
handled by \lst"xdvi.bin". 

Since correct signal handling is crucial for {\whizzy}, and this problem
is so common we provide a script to check your configuration with the command
\begin{quote}
\begin{tt}
./checkconfig
\end{tt}
\end{quote}
By default, this check is performed by automatic installation below.

\subsection {Customizing the installation}
\label {customizing}

To customize the installation, you can edit 
\lst"Makefile.config", manually.
You may also use either the command
\begin{quote}
\begin{tt}
./configure
\end{tt}
\end{quote}
This command may be passed arguments to customize your installation.
Call it with the option \lst"-help" to see a list of all options.
%
By default, the configuration is not interactive.  However, you may call it
with option \lst"-helpme" to have the script do more guessing for you and
prompt for choices if needed.

Note that by default, the Emacs-lisp code whizzytex.el is not
byte-compiled. You need to pass the option \lst"-elc" to \lst"configure" in
order to byte-compiled it.

\paragraph {Checking {\tt Makefile.config}}

A misconfiguration of your installation, or ---much more subttle--- a
misconfiguration of other commands (it appears that some installations wrap
scripts around standard commands that are sometimes incorrect and break
their normal advertized interface) may lead to systematic errors when
launching {\whizzy}. To prevent delaying such obvious errors, some sanity
checks are done after \lst"Makefile.config" has been produced and before
building other files.  These include checking for mandatory bindings (useful
for manual configuration) and for the conformance of {\tt initex}, {\tt
latex}, and viewers commands to their expected interface.

Checking viewers interface implies simulating a small {\whizzy} session: a
small test file is created for which a specializled version of latex format
is built and used to run {\LaTeX} on the test file; finally, required
viewers are tested on the DVI output, which opens windows, temporarily.

If the sanity check fails, at least part of your configuration is 
suspicious. If some windows remain opened, your confirguration is likely to
be erronesous (and so, even if not detected by the script).  

However, if you really know what you are doing, you may bypass the check by
typing \lst"make config.force", which will stamp your \lst"Makefile.config"
as correct without checking it. Checking compliance to viewers interface
is also bypassed if you you do not have a connection to X. Conversely, you
may force checking manually by typing \lst"./checkconfig".

At the end of customization, proceed as described in Section \ref
{install/automatic}.

\paragraph {Customization notes}

By default, {\whizzy} assumes the standard convention that 
\lst"latex" is the command name used to call {\LaTeX}, 
\lst"initex" the command name used to build a new format, and 
\lst"latex" is the predefined latex format.

If your implementation of {\LaTeX} uses other names, you may redefine the
variables \lst"INITEX", \lst"LATEX", and \lst"FORMAT" accordingly in the
file \lst"Makefile.config".
%
For instance, \lst"platex" could be use the default configuration
\begin{quote}
\begin{tt}
INITEX = iniptex\\
LATEX = platex\\
FORMAT = platex\\
BIBTEX = jbibtex
\end{tt}
\end{quote}
This would be produced directly with the configuration line:
\begin{quote}
\begin{tt}
./configure -initex iniptex -latex platex -format latex -bibtex jbibtex
\end{tt}
\end{quote}
If you wish to run {\whizzy} with several configurations, you must still
choose a default configuration, but you will still be able to call {\whizzy}
with another configuration from Emacs (see Section \ref{configuration.tex}
below).

It is possible to load this setup dynamically by creating, for example, 
by including the following lines: 
\begin{quote}
\begin{tt}
INITEX = iniptex\\
LATEX = platex\\
FORMAT = platex\\
BIBTEX = jbibtex
\end{tt}
\end{quote}
in a configuration file (see Section \ref {configuration}).

During the configuration, you must at least choose one default previewer
type among \lst"advi", \lst"xdvi", and \lst"ps", and at most one default
previewer for each previewer type you chose. You will still be able to call
{\whizzy} with other previewers from Emacs, via Emacs configuration (see
Section
\ref {configuration.viewers}). 

\subsection {Manual installation}

Since {\whizzy} only need three files to run, installation can also be done
manually:
\begin {itemize}

\item[]\hspace{-2em}{\tt whizzytex.el}

This could be installed in a directory visible by Emacs, but does not need
to, since you can always use the full path when you load it or declare
autoload. 

No default location.

\item[]\hspace {-2em}{\tt whizzytex}

This file is a bash-shell script that should be executable.  There is not
reason to have it visible from the executable path, since it should not be
used but with {\whizzy}.

The variable {\tt whizzy-command-name} defined in {\tt whizzytex.el} 
contains its full path (or just its name if visible from the executable
path). 

Default value is \lst"/usr/local/bin/whizzytex"

You may  need to adjust the path of \lst"bash" in the very first line of the
script, as well as some variables in the manual configuration section of the
script. 

\item[]\hspace{-2em}{\tt whizzytex.sty}

This file are {\tt latex2e} macros. There is no reason to put this visible
from {\LaTeX} path, since it should not be used but with {\whizzy}.

Variable  {\tt PACKAGE} defined in {\tt whizzytex} 
the full path (or just the name if the path is visible from {\LaTeX}. 

Default value is \lst"/usr/local/share/whizzytex/latex/whizzytex.sty"

\end {itemize}


\subsection {Automatic upgrading (depreciated)}

For convenience, the distribution also offers a facility to download and
upgrade new versions of {\whizzy} (this requires \lst"wget" to be
installed).  If automatic upgrading does not work, just do it manually.

All operations should be performed in the {\whizzy} top directory, {\em
i.e.} where you untar whizzytex for the first time, that is right above the
directory from were you made the installation. We assume that have 
created a link to the current version subdirectory: 
\begin{quote}
\begin{tt}
ln -s whizzytex-\version\space whizzytex
\end{tt}
\end{quote}
(the manager will then update this link when version changes).
Alternatively, you can also use the full name {\tt whizzytex-\version} in
place of {\tt whizzytex} below. The main commands are:
\begin{quote}
\begin{tt}
make -f whizzytex/Manager upgrade \\
make -f whizzytex/Manager install
\end{tt}
\end{quote}
The command \lst"upgrade" will successively download the newest version,
unpack it, copy the configuration of the current version to the newest
version, and bring the newest version up-to-date. The command \lst"install"
will install files of the newest version. 

The following command will (re-)install an old version:
\begin{quote}
\begin{tt}
make VERSION=<version> download downgrade install
\end{tt}
\end{quote}

%HEVEA\hypertarget{using}{}
\section{Using {\protect\whizzy}}
\label{using}

\subsection {Loading {\tt whizzytex.el}}

Maybe, {\tt whizzytex} is already installed on your (X)Emacs system, which
you may check by typing:
\begin{quote}
\begin{tt}
ESC x whizzytex-mode RET
\end{tt}
\end{quote}
If the command is understood, skip this section.
Otherwise, you should first load the library \lst"whizzytex.el" or, better,
declare it autoload. To do this permanently, include the following
declaration in your Emacs startup file (which probably is \lst"~/.emacs" if
you are using \lst"Emacs"):
\begin{quote}\small
\begin{tt}
(autoload 'whizzytex-mode \\ \indent\obeyspaces
    "whizzytex" \\ \indent\obeyspaces
    "WhizzyTeX, a minor-mode WYSIWIG environment for LaTeX" t)
\end{tt}
\end{quote}
This asumes that \lst"whizzytex.el" has been installed in your (X)Emacs
{\tt load-path}. Otherwise, you may either adjust the load-path
appropriately, or replace \lst"whizzytex" by the full path to the file
\lst"whizzytex.el", which depends on your installation and can be
obtained by typing {\tt make where} in the installation root directory.  For
instance, if you are using Emacs, the default location for
\lst"whizzytex.el" is
\lst"/usr/local/share/whizzytex/lisp/whizzytex.el" (but it will be different
if you are using XEmacs or a customized installation).


\subsection {Quick start} 

{\whizzy} runs as a minor mode of Emacs to be launched on a {\LaTeX} Emacs
buffer. The extension of the buffer should be
\lst".tex".  {\whizzy} also understands \lst".ltx" extensions, but gives
priority to the former when it has to guess the extension. Other extensions
are possible but not recommended.
\begin{quote}\em
The file attached to the buffer must exists and either be a well-formed
{\LaTeX} source file, or be {\em mastered}, {\em i.e.} loaded by another
{\LaTeX} source file. Thus, whenever the buffer does not contain a
\lst"\begin{document}" command), {\whizzy} will search for its master file,
asking the user if need be, so as to first launch itself on a buffer
visiting the master file. In particular, an empty buffer will be considered
as beeing mastered, which may not be what you intend.
\end{quote}
To start {\whizzy} on either kind of buffer, type:
\begin{quote}
\begin{tt}
ESC x whizzytex-mode RET
\end{tt}
\end{quote}
By default, this should add new bindings so that you can later turn mode
on and off with key strokes {\tt C-c C-w}. This will also add a new menu
{\tt Whizzy} in the menu bar call ``the'' menu below. (If you are using 
the {\tt auctex}, your may use other configuration key strokes to avoid
clashes (see online emacs-help). 

When {\tt whizzytex-mode} is started for the first time on a new buffer, it
attempts to configure buffer local variables automatically by examining
the content of file, and using default values of global bindings.

You may customize default settings globally by running appropriate
hooks or locally by inserting appropriate comments in the source file ---see
the manual below. 

You may also change the settings interactively using the menu, or tell
whizzytex-mode to prompt the user for confirmation of file configuration by
passing prefix argument 4 (using, for instance, key sequence 
\lst"C-u C-c C-w"). 


\subsection {Editing}

Once {\tt whizzytex-mode} is on, just type in as usual.  {\whizzy} should work
transparently, refreshing the presentation as you type or move into your
{\LaTeX} buffer. 

Additionally, a gray overlay is put outside of the current slice (the {\em
slice} is the region of your buffer under focus, which is automatically
determined by {\whizzy}). When a {\LaTeX} error occurs and it can be
localized in the source buffer, a yellow overlay also is put on the region
around the error, and removed when the error is fixed.

Furthermore, when an error is persistent for several slices or some amount
of time, the interaction-buffer will pop up with the error log
(this option can be toggled with the {\tt Auto interaction} menu entry).  

The buffer mode line also displays a brief summary of
{\whizzy}'s status.  When \lst"whizzytex-mode" is on, the line contain
$\lst"Whizzy".n$ where $n$ is a numeric indication of the load in number
of buffer changes between two slices (so the higher, the slower). 

However, $\lst"Whizzy".n$ is changed to $\lst"Whizzy-"e$ where $err$ range
over \lst"FORMAT", \lst"LATEX", or \lst"SLICE" an indicates that while
formating or \LaTeX ing the full document, or while recompiling the current
slice. Errors have priority in this order.  That is, if there is both an
error in the format and the slice, only the \lst"FORMAT" error will be
repported.

When a \lst"SLICE" error occurs, emacs attempts to locate the error
and overlay the region that caused the error. (This identifies the 
text around which  the error was detected by \LaTeX, which may not be the 
text that caused the error.) One can jump to the current error location by
calling the \texttt {Jump to error} menu emty (or the equivalent key
sequence). 


\section{Error recovery and debugging}

{\whizzy} makes a good attempt at doing everything automatically. 
However, there remain situations where the user need to understand 
{\whizzy} ---when {\whizzy} does not seem to understand the user anymore. 

\subsection {Errors while \whizzy-ing}

Quite often, the error overlay is sufficient to fix a latex source error. 
Actually, the error overaly may just indicate that you are in the middle of
typing a command or an environment, in which cases {\whizzy} will indicate
temporarily report an undefined command or and ill-balanced environment. 
Whether an overlay is ephemerous and mean an incomplete edition or 
persistent and mean a real {\LaTeX} error is usually unambiguous. 
In addition, because \whizzy ing is dynamic and the error is repported
immediately it is usually easier to fix a real error than it would be in a
batch compilation, and without even looking at the error message.

Indeed, {\whizzy} also display the {\LaTeX} error message
(and other processsing messages) in its interaction buffer.
The interaction buffer is named from the master file name surrounded
by \lst"*" characters. By default, the interaction buffer appears
in a pop up window a few seconds after an error persists and is pop down
when the error disapears. 

For serious debugging, you may unset {\tt Auto interaction} menu entry so as
to see the interaction buffer permanently. You may also unset {\tt Auto
Shrink output} menu entry to keep all log information (by default, the
interaction window is shrunk at every slice).

The {\tt View Log...} menu entry can be used to view the compele log files of
last actions performed by whizzytex (\lst"format", \lst"latex",
\lst"slice"). 

\subsection {Error during initialization}

The most delicate part of {\whizzy} is when starting {\tt whizzytex-mode},
and usually for the first time in a new buffer, since at that time all kinds
of initialization errors may occur (in addition to {\LaTeX} errors. 

{\whizzy} will attempt to identify the error and report appropriate messages
in the interaction buffer. (In case an error occurs ---or nothing happens---
always have a look at the interaction buffer first, even if it did not
prompt automatically.)

{\whizzy} keeps more debugging information during initialization phase, 
and if an error occurs during initialization, it will keep all log files.
Once initialization has succeeded  {\whizzy} turns into normal more and
by default all log and auxiliary files will be removed error et exit
(including at exit on error). However, {\whizzy} can also be launched in
debug more, which will keep additional debugging information including 
after initialization. 

To see log information, use the \lst"View log..." menu entry 
and the completion buffer. Available log files are \lst"command", 
\lst"format", \lst"latex", \lst"slice", and \lst"view".
The command log is simple the list of arguments---one per line---with which
the shell script \lst"whizzytex" was called; the log file view is the
content of the standard error description the viewer.  Some logs may not be
available if an error occured before the corresponding command has been
called. 

Most frequent errors are described below, in chronological order.


\paragraph {Emacs fails during setup}

This is the easiest case, because {\whizzy} has not been called yet, so it
is only involves debugging under emacs. 
You may check the emacs error messages (emacs buffer \lst"*Messages*"),
check the on-line documentatino of variables set or functions calls, and
in case of uncaught fatal errors, you may
\verb"ESC X toggle-debug-on-error" to get help from Emacs, and try to fix
the problem. 

Note that setup may succeed, but not be result as expected. 
You may see what configuration files have been loaded in different buffers: 
\lst"*Message*" for emacs customization, the interaction buffer 
for shell-script customozation, and the format log file for latex
configuration.


\paragraph {Emacs cannot find whizzytex}

This should typically be an installation problem, where the variable
\lst"whizzytex-command-name" is erroneous (maybe you need to give the full
path). Try to evaluate \verb"(shell-command whizzy-command-name)" in the
minibuffer, which of course should fail, but only after the command has been
reached.

\paragraph {{\whizzy} cannot build a format}

Then {\whizzy} will refuse to start. 

The problem could result from an abnormal interaction between your macros
and {\whizzy} macros, but this situation seems rather unfrequent.  So there
is most probably an error in your macros.  Try to compile {\LaTeX} your
file.  

By default the interaction window will pop-up with an section of the format
log, but you can also view the log of latex formatting 

. If this is not enough, you may need view log files.  However, log
files are normally removed when {\whizzy} exits.  To keep log files on,
you must retart {\whizzy} in debug mode (select the debug mode in the
menu and restart {\whizzy}). Then, you can check the \lst"format" log and
if necessary the \lst"command" with which {\whizzy} has been launched.
(Once the bug is fixed, you should switch off the debug mode, which may slow
down {\whizzy}.)


\paragraph {{\whizzy} cannot launch the previewer}

Usually, this is because whizzytex received wrong previewer parameter.  See
the command echoed in the interaction buffer or try to evaluate
\lst"(whizzy-get whizzytex-view-mode)".

\paragraph {Other errors}

There are two remaining problems that could happen at launch time, but that
are not particular to launch time: {\whizzy} could not recompiled the whole 
document or the current slice. However, these should not be fatal. 
In the former case, whizzytex will proceed, ignoring the whole document 
(or using the slice instead if you are in duplex mode). In the later case,
whizzytex will replace the slice by an empty slice ---and print a welcoming
document, as if you launched {\whizzy} outside of any slice. 

\subsection {Errors while editing}

After initialization time, {\whizzy} will keep recompiling slices as you
type or move, but also recompiles the format and the whole document when you
save a file. Each of this step may failed, but this should not be fatal, and
Emacs should report the error, possible pop up the interaction window, and
continue. 

\paragraph {\whizzy\space fails on the current slice}

This should not be considered as an error, it {\bf must} happen during
edition. In particular, {\whizzy} is not much aware of {\LaTeX} and could
very well slice in the middle of the typesetting of an environment or a
macro command. This should not matter, since the erroneous slice will be
ignore temporarily until the slice is correct again.

\paragraph {\whizzy\space keeps failing on the current slice}

The slice can also be erroneous because the Emacs did not correctly inferred
where to insert the cursor, which may slice erroneous, although what you
typed is correct. Hopefully, this will not occur too often, and disappear as
you move the point. It should also disappear if you switch off both {\tt
Point visible} and {\tt Page to Point} options, which is actually a good
thing to do when you suspect some misbehavior.  This will make WhizzyTeX
more robust, but less powerful and more boring.

\paragraph {\whizzy\space does not seem to slice at all}

The interaction window does not produce any output. 
Try to move in the slice, or to another slice. 

If nothing happens, check the interaction
window, to see if it did attempt to recompile the slice.
If nothing happens in the interaction window, check for Emacs messages
(in the \lst"*Messages*" buffer). You may also check for the presence 
(and content) of the slice by visiting 
\lst"_whizzy_filename.tex" or
\begin{quote}
\begin{verbatim}
_whizzy_filename/input/_whizzy_name.new
\end{verbatim}
\end{quote}
If neither file exists, it means that Emacs did
not succeed to slice, which you may force by evaluating
\lst"(whizzy-observe-changes t)". 
This can be run in even if {\tt whizzytex-mode} is suspended, which may
avoid automatic processing of slices, and their erasure.

If the slice is present, you may try to compile it by hand (outside of
Emacs) with 
\begin{quote}
\begin{verbatim}
latex '&_whizzy_filename' _whizzy_filename.tex
\end{verbatim}
\end{quote}
and see if it succeeds. 


\paragraph {Reformatting failed}

Formatting errors are fatal during initialization, but accepted once
initialized. In case of an error during reformatting, {\whizzy} will ignore
the error and continue with the old format.  This means that new macros may
be ignored leading to further slicing errors. When rebuilding the format
failed, the mode-line string will display the suffix \lst"FMT" until the
error is fixed.  See the interaction buffer or select \lst"format" from the
\lst"log..." menu entry).

You may also force reformatting by typing the \lst"reformat" command
in the interaction buffer. 


\paragraph {Whizzytex cannot process the whole document}

This is very likely a problem with you document, so try to {\LaTeX} it 
first. There is a small possibility of strange interaction between
your macros and {\whizzy} package. Try to turn options 
{\tt Page to Point} and {\tt Point visible} off and retry. 
This will make {\whizzy} more robust (but also less powerful and more
boring). 

\subsection {Debugging}

If you are still completely lost after trying all of the above help, you may
turn on the debugging mode by typing either line in the interaction window:
\begin{quote}
\begin{verbatim}
trace on
trace off
\end{verbatim}
\end{quote}
or with the menu entry \lst"Debug".
The entry can also be called to start {\whizzy}, which will then start in 
debugging mode, including during initialization. 

If need be, you can also turn emacs debug mode on and off with
\begin{quote}
\begin{verbatim}
ESC x toggle-debug-on-error RET
\end{verbatim}
\end{quote}

If you are still stuck, then you are left on your own and need real
debugging.  If this is your first attempt at {\whizzy}, you should suspect
your global configuration. You should then try it first with the examples of
the distribution.  Otherwise, you may rollback to a file and configuration
that used to work (e.g. one of the distribution), and make incremental or
logarithmic changes until you hit the problem. 

%HEVEA\hypertarget{help}{}
\section{On line help}

The Emacs source is fully documented and most of the documentation is
available  as on-line Emacs help, through the \lst"Help" entry of the
\lst"Whizzy" menu and following hyperlinks.
%
Alternatively, you can type
\begin{quote}
\begin{verbatim}
ESC x describe-function RET whizzytex-mode RET
\end{verbatim}
\end{quote}
(In XEmacs, you may need to use
\begin{quote}
\begin{verbatim}
ESC x hyper-describe-function RET whizzytex-mode RET
\end{verbatim}
\end{quote}
instead of \lst"describe-function" to see hyper-links.)

To avoid redundancy, on-line help is not reproduced here, configuration
described in the next section.


%HEVEA\hypertarget{manual}{}
\section{Configuration} 
\label{manual}

This section describes how to use and parameterize {\whizzy}.  
Section \ref{configuration}, \ref{modes} and
\ref{types} are also available as online help. 


\subsection {Emacs global configuration}

\label{configuration.viewers}
\label{configuration.bindings}
\label{Emacs-configuration}

See Emacs help for \lst"whizzy-default-bindings" and
\lst"whizzytex-mode-hook" for list of bindings.

The Emacs on-line help for \lst"whizzytex-mode" lists all user-configurable
variables,  which  may be given default values in your Emacs startup file
to be used instead of {\whizzy} own default values. 

\subsection {File-based configuration}

\label{configuration}
\label {File-configuration}

{\whizzy} allows for inlined customization in the source file, as described
below. While this mecanism is quit convenient for short and simple
customization (such as selecting the output format and previewer or
sectioning), it is harsh and \emph{depreciated} for advanced customization,
for which you should prefer local customozation files (see Section~\ref
{sec/local-custom}). 

A configuration line is one that starts with regexp prefix ``\lst"^%; +"''
followed by a configuration keyword.  If two configuration lines have the same
keyword, only the first one is considered. The argument of a configuration
line is the rest of the line stripped of its white space.

The keywords are:
\begin{description}
\item[whizzy-master]\arg {master}
\\
This only makes sense for a file loaded by a {\em master} file. 
\arg{master} is the relative or full name of the
master file. Optional surrounding quotes (character \lst`"`) %" 
%
stripped off, so that \lst$"foo.tex"$ and \lst"foo.tex" are equivalent.

\item[whizzy-macros]\arg {master}
\\
  This is equivalent to {\bf whizzy-master} \arg{master}, but for a file
  containing   macros. The file is not sliced while editing, but 
  saving it reformats the master. 

\item[whizzy] 
\opt{\arg{slicing}} 
\opt{\arg{viewer} \opt{\arg{command}}}\\
\opt{\texttt{-mkslice} \arg{command}} 
\opt{\texttt{-mkfile} \arg{command}} \\
\opt{\texttt{-tex} \arg{suffix}} 
\opt{\texttt{-initex} \arg{initex}}
\opt{\texttt{-latex} \arg{latex}}
\opt{\texttt{-fmt} \arg{format}}\\
\opt{\texttt{-bibtex} \arg{bibtex}}
\opt{\texttt{-dvicopy} \arg{command}}
\opt{\texttt{-watch}}
\opt{\texttt{-duplex}}
\opt{\texttt{-trace}}
\\[1em]
All arguments are optional, but if present they must appear in order and on
a single line:
\begin{description}
\item[\arg{slicing}]\indent\\ 
determines the way the document is sliced
(see section~\ref{modes}).

\item[\arg{viewer}]\indent\\
is the type of viewer and can only be one of 
\lst"-advi", \lst"-xdvi", \lst"-ps", or \lst"-pdf" (see section~\ref{types})

\item[\texttt{-display} \arg{display}]\indent\\
specifies which X display to show the DVI previewer in,
such as \texttt{:0.1} for multidisplay set-ups.

\item[\arg{command}]\indent\\
is optional and is the command used to call the viewer
(of course, it should agree with \arg{viewer}). 

\item[\texttt{-mkslice} \arg{command}]\indent\\
tells {\whizzy} to use  \arg{command} to preprocess the slice. 
The command \arg{command} will receive only one argument
which is the slice to be produced \texttt{\_whizzy\_basename.tex} 
(or \texttt{\_whizzy\_basename.ltx} if the extension of the master file is 
\texttt{.ltx}) from the original \texttt{\_whizzy\_basename.new} 
(whose name is obtained by replacing the suffix ".tex" or ".ltx" by ".new")
By default, the command \lst"mv" is  used to move the original to the
destination. (See the example in \lst"./examples/gpic/")

{\em The Unix \lst"make"  can itself be used as a preprocessor (with an
appropriate \lst"Makefile").  However, one may have to work around
\lst"make"'s notion of time (using FORCE), which is usually too rough. 
This is safe, since {\whizzy} tests itself for needed recompilations.}

\item[\texttt{-mkfile} \arg{command}]\indent\\
executes ``\arg{command} \arg{filename}'' before recompiling every time a
buffer is saved. The argument ``\arg{filename}'' is the buffer-file-name
path relative to the path of the master file directory.

\item[\texttt{-makeindex} \arg{command}]\indent\\
uses ``\arg{command} \arg{filename.idx}'' for rebuilding the index instead
the default ``\arg{makeindex} \arg{filename.idx}''.  If ``\arg{command}'' is
false, then do not attempt to rebuild the index.

\item[{\bf {\tt -bibtex \arg{bibtex}}}]\indent

uses \arg{bibtex} for the bibtex command instead of the value 
assign to BIBTEX in \lst"Makefile.config" (or \lst"whizzytex")

\item[{\bf {\tt -initex \arg{initex}}}]\indent

uses \arg{initex} for the initex command instead of the value 
assign to INITEX in \lst"Makefile.config" (or \lst"whizzytex")

\item[{\bf {\tt -latex \arg{latex}}}]\indent

uses \arg{latex} for the latex command instead of the value 
assign to LATEX in \lst"Makefile.config" (or \lst"whizzytex")

\item[{\bf {\tt -fmt \arg{format}}}]\indent

uses \arg{format} for the latex format instead
of the default value, usually fmt (see configuration).

{\em This can either be used in combination with \lst"-latex" and
\lst"-initex", 
or alone. For instance,
\lst"hugelatex" could be used (depending on your {\LaTeX} configuration) to
build a larger format to process huge files.}
\label{configuration.tex}

\item[{\bf {\tt -dvicopy \arg{command}}}]\indent
\label {sec/dvicopy}

uses \arg{command} instead of the default (mv) to copy DVI files
(from \lst"FILE.dvi" to \lst"FILE.wdvi"). This can be used with command
\lst"dvicopy" so as  to expand virtual font, which advi does not understand
yet) 

\item[\texttt{-watch}]\indent\\
watches other files than just the slice (see Section~\ref {sec/watch}).

\item[\texttt{-duplex}]\indent\\
launches another window with the whole document (which is
recompiled every time the source buffer is saved).

{\em With \lst"-advi" previewers, both views  communicate with Emacs and can be
used to navigate through source buffers and positions.}

\item[\texttt{-trace}]\indent\\
traces all script commands (for debugging purposes only.)
\label{configuration.trace}

\end{description}

For instance, a typical configuration line will be:
\begin{verbatim}
   %; whizzy subsection -dvi "xdvi -s 3"
\end{verbatim}
It tells whizzytex to run in subsection slicing mode and use a \lst"dvi"
style viewer called with the command
\lst"xdvi -s 3". This is also equivalent to
\begin{verbatim}
   %; whizzy subsection -dvi xdvi -s 3
\end{verbatim}
since Emacs removes outer double-quotes in option arguments. 

A more evolved configuration line is:
\begin{verbatim}
   %; whizzy -mkslice make -initex iniptex -latex platex -fmt platex
\end{verbatim}
It tells {\whizzy} to use \lst"iniptex" and \lst"platex" comands instead
of \lst"initex" and \lst"latex" and to use the format file \lst"platex.fmt" 
instead of \lst"latex.fmt". Moreover, it should use \lst"make" to preprocess
the slice.

\item[whizzy-paragraph] \texttt{regexp}\\
This sets the Emacs variable \lst"whizzy-paragraph" to \texttt{regexp}.
\end{description} 


\subsection {Modes} 
\label {modes}

{\whizzy} recognizes several modes,
\lst"slide",
\lst"section",
\lst"subsection",
\lst"paragraph",
\lst"document", and
\lst"node", described below.

The mode determines the slice of the document being displayed and indirectly
the frequency of slicing.

Note that in any mode but \lst"none" slices are always included in the file
beeing editing or other files included.  When slice delimitors are not
found, the slice default to the whole file. The slice may also be empty if
the cursor is located before \lst"\begin{document}" or after
\lst"\end"\lst"{document}".

\begin{description}

\item [slide]

The mode \lst"slide"  is mainly used for documents of the class seminar. 
In slide mode, the slide is the text between two \lst"\begin {slide}"
comments (thus,  the text between two slides is displayed after the
preceding slide).  

In slice modes, overlays are ignored {\em i.e.} all overlays all displayed in
the same slide, unless a command
\lst"\overlay {"$n$\lst"}" occurs on the left of the point, on the same line
(if several commands are on the same line, the 
right-most one is taken), in which case only layers $p \le n$ are displayed.

\item [section]
In \lst"section" mode, the slice of text is the current chapter, section.

\item [subsection]
As \lst"section" but also slice at subsections. 

\item [paragraph]
The \lst"paragraph" mode is a variation on section mode where, the separator
\lst"whizzy-paragraph" is defined by the user (set to two empty lines by
default) instead of using \lst"\section"  and \lst"\subsection" commands. 
subsection.

\item [document]
The \lst"document" take the region between \lst"\begin{document}"
and \lst"\end"\lst"{document}" as the slice. Hence it defaults to the file
when the file is a slave, which does not contain \lst"\begin{document}". 

\item [none]
In \lst"none" slicing mode, there is no sectioning unit at all and
the whole document is recompiled altogether. 
Currently, pages are not turned to point and the 
cursor is not shown as in \lst"document" mode, because full documents are not
sliced. (A slicing document mode could be obtained by working in paragraph
mode, with an appropriate regexp.)

\end{description}

\subsection {Viewer types}
\label {types}

See help for \lst"whizzy-viewers".

The previewer types can have three possible values:
{\tt -advi}, {\tt -dvi}, {\tt -ps}, or {\tt -pdf}. 

The previewer type should agree with the previewer command in several ways:
\begin {itemize}

\item
They tell how to trigger reload on the previewer. 
This may signal the previewer with signal \texttt{SIGHUP} for \texttt{-ps}
or {\tt SIGUSR1} for \texttt{-dvi} and \texttt{-advi}, or to establish the 
previewer as a remote server with \texttt{-pdf}.


In particular, if you write a front-hand shell-script \lst"viewer" to the
call previewer,  and want to use \lst"viewer" as the previewer, you should 
arrange for \lst"viewer" to understand these signals (and forward them to the
previewer). The simplest way is to hand your script with an exec command
calling the \lst"gv", \lst"dvi" or \lst"advi".

Also, the option \texttt{-pdf} assumes \texttt{xpdf} remote server 
(launched with the whizzytex process id as name) and its reload protocol. 
Thus, if you wish to use another previewer, you also need to cutomize the
variable \texttt{RELOAD} of the shell-script. 

\item
They tell whizzytex whether to process the slice to 
Postscript (with {\tt -ps}) or to DVI format (with {\tt -dvi} and {\tt
-advi} or directly generate \texttt{pdf} output with \texttt{pdflatex}.

\item
Moreover, {\tt -advi} requires the previewer to 
recognize additional \lst"\special" commands, in particular
source line information of the form: 
\begin{quote}
\begin{verbatim}
#line 780, 785 <<to<<rec>><<ognize>>additional>> manual.tex
\end{verbatim}
\end{quote}

\end {itemize}
Then, the previewer command is the command to call the previewer.  This
string will be passed as such to the {\whizzy} shell-script. Note that the
name of the \lst"dvi" or postscript file will be appended to the previewer
command.

\subsection {Configuration via configuration files}

This is preferable to file-based configuration.
{\whizzy} also recognizes configuration files 
\lst"whizzy.sh", \lst"whizzy.sty", \lst"whizzy.el" 
in either  your home directory or the current working directory. 


File \texttt{whizzy.sh} is read by the script \texttt{whizzytex} after some
defaults values (usually determined by the configuration) have being
assigned to variables. Hence, it can be used to redefined those values. 

Here are typical default values that can be redefined. 
$$
\begin{tabular}{>{\tt}l>{\tt}l>{\tt}ll}
\rm \bf Variable & \rm \bf default value & \rm \bf other value
& \bf Meaning
\\[1ex]
LATEX & latex & elatex & latex program 
\\
FORMAT & latex & elatex & format basefile name
\\
FMT & fmt & efmt & format extenstions
\\
DVICOPY & mv & dvicopy & transformation $f$\texttt{.dvi} file into $f$\texttt{.wdvi}
\\
BIBTEX & bibtex & ebibtex & bibtex  program
\\
MKSLICE & defaultmkslice & . & hook for preprocessing slices
\\
MKFILE & defaultmkfile & . & hook for preprocessing files
\\
MAKEINDEX & makeindex & . & command for making index
\\
\end{tabular}
$$
See \texttt{whizzytex} for other bindings. 
Other shells environment values such as \texttt{TEXINPUTS} may be defined
here as well.

File "whizzy.el" is read by emacs when turning the whizzytex mode on for the
first time.  File "whizzy.sty" is read by latex every time the document is
reformatted.

You may also use files 
$\langle masterbasename\rangle$\lst".wsh", 
$\langle masterbasename\rangle$\lst".wel", 
$\langle masterbasename\rangle$\lst".wsty", 
in the current working directory to only load the file when
running whizzytex on the master file $\langle masterbasename\rangle$. 





\subsection {Watching other files}
\label {sec/watch}

{\whizzy} is designed to watch other files and not just the slice saved by
Emacs. In fact, it watches any file dropped in the pool directory. 
For instance, 
if your source file uses images, you can just change the image and
drop the new version in the pool. Then {\whizzy} will pick the new version,
move it to the working directory and recompile a new slice. Be aware of name
clashes: if you drop a file in the pool, it will automatically be move to
the working directory with the same name, overriding any file of the same
name sitting there. 

However, activity is entirely controlled by Emacs, since after every
iteration {\whizzy} waits for Emacs to send a new command (usually the empty
command that means iterate again). Hence, other files will only be taken
into account at the next iteration. If you really wish these files
to be watched you need to instrument emacs to send and empty line input to
the interaction buffer regularly, even when idle. 

\subsection {Frequency of recompilation} 

To obtain maximum {\whizzy} effect, a new slice should be save after any
edition changed or any displacement that outside of the current slice.
However, to avoid overloading the machine with useless and annoying
refreshments, some compromise is made, depending on Emacs several
parameters: edition {\em v.s.} move Emacs last commands, 
successful {\em v.s.} erroneous last slice, and the duration of last slice
recompilation. This usually works well. However, different behavior may wish
to be obtained in different situations. For instance, when editing on a
lab-top, one may wish to save batteries by keeping the load rather low, hence
not using the full power of the processor. Conversely, one may wish 
{\whizzy} to be as responsive as possible. There is an function
\lst"whizzy-load-factor" that control a variable of the same name, which can
be used to adjust the responsiveness (by increasing or decreasing the
load-factor). This simply adds extra delays between slicing. 

The format is automatically recompiled at the beginning of each session, and
whenever the buffer containing the file is saved. That is, to load new
packages or define new global macros (before the \lst"\begin{document}"), it
suffices to save the current file.

\subsection {\whizzy-ing macro files} 

Macro files can be \Whizzy-ed as well.  The effect is them only to
automatically call \lst"reformat" when the file is saved.
Files can also be declared as macro-files with 
\lst"whizzy-macro" file configuration keyword (see Section \ref
{File-configuration}), which argument should then indicate the master file.
Files with \lst".sty" extension are by default considered as macro files
and their master file is guessed if possible.


\subsection {Cross-references, page and section numbers} 

The slice is always recompiled with the \lst".aux" file of the whole
document.  In paragraph mode, cross references and section numbers are 
recompiled whenever the buffer itself is saved (manually). 
The recompilation of the whole document is off in slide mode.

\subsection{Per session {\LaTeX} customization}

The Emacs variable \lst"whizzy-customize" (that can be set
interactively from the \lst"Customize slice" menu) may contain
a few {\LaTeX} commands to be inserted at the beginning of each slice, which 
allows a per-session customization.  Customization can be easily changed
anytime in the middle of a session. For instance, setting this variable to
\lst"\large" can be used to temporarily enlarge the text, while keeping the
same page layout.


%HEVEA\hypertarget{custom}{}
\subsection{System, user, and local customization} %
\label{sec/local-custom}

{\whizzy} is a three-part engine, with Emacs, Latex, and the glue
Bash-script running altogether. Some of the parameters can be adjusted at
installation-time by modifying the respective files \lst"whizzytex.el",
\lst"whizzytex.sty", or \lst"whizzytex" of the distribution.  However, you
should normally not have to do that after installation (and even only
exceptionnally during installation), and instead use system, user, or local
configuration files.

When launched, each engine looks for configuration files in appropriate
directories with basenames \lst"whizzy.el", \lst"whizzy.sh", 
\lst"whizzy.sty" and \lst"whizzy.cfg", respectively. The Emacs configuration
search path is defined by the emacs variable variable
\lst"whizzy-configuration-path".  Search path for Bash and Latex settings
are composed of the directories \lst"CONFIGDIR/", \lst"$HOME/.whizzytex/"
and the current directory (actually \lst"$TEXINPUTS" for latex).  All
configuration files found are loaded, in the order given above.
The difference between \lst"whizzy.sty" and \lst"whizzy.cfg" is that the
former is loaded after all latex packages and typically use to change the
behavior of those packages while the later is loaded first,
before \lst"\documentclass" and is rather use used to redefined
commands of "whizzytex" or defined extra macros before loading the
document (e.g. as arguments that could normally be defined on the
commandline while calling \lst"latex").

Remark that a local configuration file (\emph{i.e.} one in the current
directory) can be used to make per-document configuration by testing on 
jobname.

\iffalse
\subsection*{Example of {\LaTeX} per-document customization}

\whizzy attempts to make the slice appears as much as possible as in 
the full document, so that all typesetting adjustments, including overfull
boxes, {\em etc.} can be fixed under \whizzy. 
(Some tricks, such as showing the cursor, may disturb the
presentation, but locally.)

Still, one may wish to permanently or temporarily give the document another
appearance when under \whizzy. For instance, you may wish to switch to draft
mode under \whizzy (maybe using a few {\ActiveDVI} tricks to include
annotations) and use a normal mode under regular processing.

This can be achieved by defining macros in a file \lst"whizzy.sty" in the
{\LaTeX} path that will be automatically loaded by 
{\whizzy} at the end format building. This file can be be placed in your
macro directory and be used for all documents (you may thus customize
{\Whizzy} itself), or in the current directory
and be used for only some documents.
\fi

%HEVEA\hypertarget{viewers}{}
\section{Viewers}


\subsection {Viewing with \ActiveDVI}

\href{http://gallium.inria.fr/advi/}{\ActiveDVI} is a DVI previewer with
several additional features. 
In particular, it recognizes extra specials, some of which are particular 
useful for whizzytex that allows a two way communication between 
the source Emacs buffer and the previewer: 
\begin {itemize}
\item
The previewer will automatically turn pages for you, as you are editing. 
This is done by telling Emacs to save the current position in the slice. 
Then, the recompilation of the slice will include the current position 
as an hyperref location \lst"Start-Document" whenever possible. 
Then, just tell {\ActiveDVI} to automatically jump at this location
when it opens/reloads the file (option -html Start-Document).

\item
Conversely, {\ActiveDVI} can dump source file positions on clicks, 
when available (usually on \lst"shift-mouse-1" or \lst"mouse-1" in
\lst"edit" mode), that
is forwarded to Emacs so that it can move to the corresponding line.

To enjoy this feature, the option \lst"-advi" should be used instead of
\lst"-dvi". This will produce extra information (such as source line
numbers) using \lst"\special" that most DVI previewers do not recognize
and may complain about.

\item
{\ActiveDVI} does not currently recognizes virtual fonts, but \lst"dvicopy" 
can be used to expand them. See the option \lst"-dvicopy" in Section~\ref
{sec/dvicopy}. 

\item
\noindent
\NEW{1.1.3}\relax
If you have a recent version of {\ActiveDVI} (version number exists and is
greater than 1.5.2), you can also enjoy the multiple view mode, which is
configured by default (variable \lst"MULTIPLE" is set to \lst"true" in
Makefile.config). In this case, {\Whizzy} will call the previewer both
the slice and the whole document in the same window and may automatically
switch from the slice to the whole document when clicking on local hyperrefs
that are out of the slice (press \lst"Esc" to come back). You can also
switch between views by pressing \lst"w" and when on the whole document
view, goto the page when the cursor is in Emacs by pressing \lst"W". 

\emph{Warning! If by mistake or misconfiguration, the multiple view is
enable and your version of advi does not support multiple views, you
will only see the full document view and never see the slice. }
\end {itemize}

\subsection{Defining your own previewer}

To use your own command as a previewer, you must choose either type
\lst"-dvi" or \lst"-ps" . In particular, your previewer should 
accept \lst"SIGUSR1" (for \lst"-dvi") signal or \lst"SIGHUP" (for \lst"-ps") 
signal and respond by reloading the file.

%HEVEA\hypertarget{pdf}{}
\subsection{Viewing with \texttt{xpdf}}


{\whizzy} also works with \lst"pdf" using the \lst"xpdf" previewer
and its remote server capabilities to reload the file and jump to the
cursor position (this does not work with \lst"acroread" because they is no
simple way to tell 
\lst"acroread" to reload its file in batch). You must choose \lst"-pdf" 
as previewer type, which will also set other variables so as to compile the
document with pdflatex instead of latex. You must leave the default
previewer command, i.e. enter \lst"-pdf ." and not \lst"-pdf xpdf" (or else
understand the internals of the \lst"whizzytex" script) because
other options need to be passed to \lst"xpdf".

When compiling with \lst"pdflatex", you need to explicitly use the package
\lst"hyperref". 

Another solution is to compile via the generation of a Postscript file
\lst"ps" and its translation to PDF using \lst"ps2pdf". 
This allows the use of \lst"pstricks". In this case, you tune the
configuration manually, following the example given in \lst"examples/ps2pdf/". 


\section{Whizzy Effects}

Since {\whizzy} knowns about the current point in the buffer, rendering
of the document may depend on that possition. For examples, an environment
may be displayed differently when the point is inside or outside the
environment. A natural choice is to make drawer-like environments that 
are \emph{closed} when the point is outside and \emph{open} when the point
is inside.

{\whizzy} provides a the macro \lst"\WhizzyInsideEnvironment"
to help make such effects. It takes the same parameters as the command
\lst"\newenvironnement". The first argument should be the name of an
existing environment, which will behave as before when the point appears
outside and according to the new definition when the points is inside.  The
second and first arguments defines the behavoir as do the arguments of
\lst"\newenvironment". However, \lst"\WhizzyInsideEnvironment" also defines
the macro \lst"\out@myenv" and \lst"endout@myenv" to refers to the
cursor-outside version of the environment. Typically, these macros can be
used in the second and third argument of \lst"\WhizzyInsideEnvironment" to
define the cursor-inside version by difference with the cursor-outside
version. 

The example \lst"effects" shows two applications.  First, a \lst"drawer"
environment is used to delimit sections and make them open or closed
automatically as cursor moves. Second, using the
\texttt{exercise} package, we provide a cursor-inside version of the answer
environment that inline the answer rather than pushing it to the Appendix. 


%HEVEA\hypertarget{edit}{}
\section {\WhizzyEdit ing}
\begin{quote}
\em
\noindent\NEW{1.1.3}\relax
This feature requires at least version \texttt{1.60} of
{\ActiveDVI}.
\end{quote}

When used together with Active-DVI, {\whizzy} can be made much mode
powerful. In particular, it is not difficult to lift  {\whizzy} from an
incremental viewer to an assistant editor.

What was a dream has now become real.
%
The latest version Active-DVI provides a notion of active boxes. 
The DVI may be  annotated with \verb"advi: edit" specials commands.  
When {\ActiveDVI} is put in edition mode,
active boxes are drawn on top of the previewer window  and can be move or
resized with the mouse. 
When the mouse is released, the new size or position is printed on standard
output together with the action to be taken and received by emacs watching
the output.
%
Emacs has then enough information to adjust some dimensional parameters in
the source buffer. Just after this edition, the new slice is processed and
the new position is displayed.  Thanks to the short incremental loop, this
almost appears as if actions where executed by Active-DVI itself.

Indeed, \WhizzyEdit ing is not meant to break up the structual edition
philosophy of {\TeX} and {\LaTeX}. Its incremental viewing is an assistant
to an not a replacement of structural source edition. Mouse editing should
also be seen similarly.  In particular, all editions are visible in the
emacs source buffer, can be saved, manually changed or disable. Moreover,
Whizzy-editing is not meant for document layout (even it can occassionally
be used for that, \emph{e.g.} in slides), but rather to help adjust
dinmensions that require manual tuning.

For instance, imaging you are importing an Encapsulated Postscript picture
you would like to place some bubble whose origin must be position precisely
inside the picture. Then, you'd better do it with the mouse rather than by
small measurements or adjustements. Drawing a graph with a few nodes may now
become quite confortable with PStricks, with the advantage of remaining
within {\LaTeX} rather than using some external tool. Finally,
Whizzy-editing is likely to be convenient when writting slides with visual
gadgets. For instance, adjusting bubbles with the mouse is likely to be more
efficient than doing it by hand.


\subsection{Enabling edition with the \texttt{\protect\string\adviedit} macro}

{\ActiveDVI} provides one general editing command that can be used by
{\Whizzy} for all mouse editing.  The syntax of this command is
\begin{quote}
\docdef \adviedit{[tag]}\docarg{options}\docarg{body}
\end{quote}
where \docid{options} is a comma separated list of bindings
according to the \doctt{keyval} package. Each binding is either of the form
\docid{var}=\docid{float} where 
\docid{var} ranges other letters 
\doctt{x}, \doctt{y}, \doctt{h}, \doctt{w}, \doctt{d} 
in lowercase or uppercase, or \doctt{field}=\docid{dimension} where 
\docid{field} ranges over \docid{unit} and \docid{min}.

The \docid{field} respectively bindings specifies the unit, which default to
\doctt{1em}, and the minimal dimension of boxes. Both fields are inherited,
which enable inner edition to be scale altogether.  The \docid{var} bindings
defines values for the corresponding variables. 
The are not inherited. On the opposite, they are always reset to
default values.  Lowercase letters mean that the corresponding variables
are whizzy-editable, while uppercase letters treat them as constants. 
The expression \doctt{body} should be horizontal box material: it is then
placed in an \doctt{\string\hbox} at coordinates (\doctt{x}, \doctt{y})
relatively to the current position.  
Moreover, a virtual box of width \doctt w, height \doctt
h, and depth d is draw at that position when editing is made active.
The box can this float around the current point and has no dimension.
However, a box with no coordinates specified is fixed and has the dimensions 
of \doctt w, \doctt h, and \doctt d. When not specified, these fields takes
the value of the box in which body is typeset. 
All dimensions \doctt x, \doctt y, \doctt w, \doctt h, and \doctt d are
bound to \doctt{advix}, \doctt{advix}, \doctt{adviy}, \doctt{adviw},
\doctt{advih}, and \doctt{advid} macros during the evaluation of
\docid{body}. 

Whizzy-editable objects can be nested. All parameters are reset to default
values, within the new object. Sometimes, emacs may be confused and take an
object for another. In these rare cases, the two objects can use the 
\docid{tag} argument to be distinguished. This argument does nothing but
being passed to {\ActiveDVI} and sent back to Emacs to identified 
the object exactly. 

\subsection{Performing mouse edition under {\ActiveDVI} control} 

\begin{quote}\em
This section depends entirely on {\ActiveDVI}. Hence, it may depend on your
version of {\ActiveDVI} or how you have parameterized it.  The appearance
and description below is based on default bindings for version
\texttt{1.50+3}.
\end{quote}

To actually \emph{edit} whizzy-editable objects, you need to toggle the
\emph{edit} mode of {\ActiveDVI}. You can do this interactively by key stoke
\lst"e" in the {\ActiveDVI} window. You may also start {\ActiveDVI} in
\emph{edit} mode by passing the option \texttt{-edit}.

When in edit mode, whizzy-editable objects are drawn as below:
\begin{center}
\begin{toimage}
\psframebox[framesep=2em]{\adviedit
{w=0.7571,unit=\hsize}{\parbox[t]{\adviw}{\setedit{unit=\adviw}\relax  
This small page contains

\hbox {\hbox to 0em
{\hss\circlenode[linecolor=green,framesep=0.4ex]{A}{}\hss}%
\adviedit{x=0.3418,y=-0.1445,w=0.6574}
{\pnode{P}\psframebox[linecolor=blue,framesep=0pt]{\setbox0=\hbox{\parbox[c]{\adviw}{This
whizzy-editable parbox drawn in a blue rectangle. 
The blue line in the middle of the rectangle marks the baseline of the box. 
The green line is attached to the left hand-side of the baseline on the one
hand  and to the anchor of this object, dran as a small green circle, on the
other hand. 
}}\box0\pnode{Q}%
\ncline[linecolor=green]{A}{P}%
\ncline[linecolor=blue]{P}{Q}%
}}}%
and right

below is\ldots

\setbox0=\hbox{\adviedit{d=0.1780}{\vtop{\vspace{\advid}}}}%
 \wd0=1ex\ht0=1ex
\psframebox[linecolor=blue,framesep=0pt]
{\pnode{P2}\hbox to 0em
{\hss
\circlenode[linecolor=green,framesep=0.4ex]{A}{}
\hss}%
\box0
\pnode{Q2}\ncline[linecolor=blue]{P2}{Q2}}%

a whizzy-editable vertical space. 

\hspace{1.5em}%
The vertical space is an unmovable resizable box whose dimensions are as
displayed in blue.  
In contrast, the parbox above is movable (and width-resizable) and behaved
as an empty box. }}}
\end{toimage}\imageflush
\end{center}
\par\noindent
You may edit such objects in two ways: 
\begin{itemize}
\item \textbf{move} them, using the middle button.
\item \textbf{resize} them, using the right button for width and height
or the shift-right button for depth.
\end{itemize}
When pressing the button on the corresponding rectangle, the mouse shape
should intuitively illustrate the action to be perfomed. 
However, some actions may be inhibited. For instance, the
\texttt{\string\parbox} can only be moved or resized in width and the
vertical space can only be resized in depth but not be moved.  When an
action (either \emph{move} or \emph{resize}) is disable in all directions,
the cursor will not changed. When resizing is enabled both in \emph{height}
and in \emph{depth}, the default action is \emph{height} and you must press
the shift key to perform the \emph{depth} resizing.

Finally, an edition can be aborted by pressing the \emph{meta} key (actually
the one bound to \emph{modifier-1}) while release the mouse. 


\subsection {Examples}
Several examples can be found in file {\tt example/edit/main.tex} coming with
{\ActiveDVI} distribution. Here are a couple of simple ones. 
For example, 
\begin{verbatim}
        \adviedit{x=-2.8845,y=0.2717}{A}
\end{verbatim}
will simply place make the letter $A$ whizzy-movable. 
The values of $x$ and $y$ when unspecified defaults to $0$. 
Values for $W$, $H$ or $D$ when not given, will default to the
value  of $\hbox{A}$. However, if $W$, $H$, or $D$ are zero (or too small)
they will default to some small value.
\begin{verbatim}
        \adviedit{X=2,Y=3}{A}
\end{verbatim}
can simply be used instead of the latex $\string\put$ command.
Spaces are also whizzy-adjustables: an horizontal space is just
\begin{verbatim}
        \adviedit{w}{\hspace{\adviw}}
\end{verbatim}
Note that the material is placed into a default \texttt{\string\hbox}. 
Thus, for vertical spaces, one need and explicit \texttt{\string\vbox}:
\begin{verbatim}
        \adviedit{d}{\vtop {\vspace {\advid}}}
\end{verbatim}
Note that 
\begin{verbatim}
        \adviedit{h}{\vbox {\vspace {\advih}}}
\end{verbatim}
would do as well, but would usually be less intuitive, graphically.

A paragraph of adjustable size: 
\begin{verbatim}
        \adviedit{w}{\parbox[c]{\adviw}{text material}}
\end{verbatim}
Whizzy-edition can also be used to resize images (as well as return them)
\begin{verbatim}
        \adviedit{w,h}{\includegraphics[width=\adviw,height=\advih]{caml.eps}}
\end{verbatim}
Note that while {\string\adviedit} must remain in the should, hence the
whole line cannot be abbreviated into a macro, one can freely abbreviate its
body, and it is quite easy to build a camel caravan: 
\begin{verbatim}
        \adviedit[A]{w,unit=\hsize}{%
          \setedit{unit=0.2\adviw}%    
          \def \camel{\includegraphics[width=\adviw,height=\advih]{caml.eps}}%
          \adviedit{x,y,w,h}{\camel}%
          \adviedit{x,y,w,h}{\camel}%
          \adviedit{x,y,w,h}{\camel}%
          \adviedit{d}{\vtop{\vspace\advid}}% 
          \hspace{\adviw}% 
        }
\end{verbatim}
Be aware that a camel may hide another one! Indeed, at the beginning all
camels are superposed. The first caml you pick is the one in front. 
An interesting use of units is to let an inner editable command sets its
unit according to the dimension of an outer command, as illustrated above.
Here the outer object (tagged \texttt{A}) is used to control the origin
and scale of the projection.  Then, each camel can be translated and
resized, but relatively to this origin and this scale. Thus moving 
of rescaling the outer object will treate the caravan as a whole.
The last line allow expansion of the bounding box as needed. 
The one before last sets the vertical ratio of the bounding box.
The result can be seen in Figure~\ref {caravan}. 
\begin{figure}
\centerline{%
\begin{toimage}
\epsbygs
\def \camel{\includegraphics[width=\adviw,height=\advih]{caml.eps}}
%%To adjuste the scale...
\adviedit{w=0.5633,unit=\hsize}{%
\setedit{unit=0.2\adviw}%
\adviedit{x=3.8029,y=-0.9056,w=-0.5217,h=0.8348}{\camel}%
\adviedit{x=4.2886,y=-3.1835,w=-1.6228,h=2.1934}{\camel}%
\adviedit{x=1.1410,y=-0.9226,w=1.1661,h}{\camel}%e
\adviedit{x=0.1224,y=-1.5206,w=1.4992,h=1.3165}{\camel}%
\adviedit{x=1.4281,y=-1.5888,w=1.5068,h=1.3165}{\camel}%
\adviedit{x=0.5156,y=-3.1263,w=1.3621,h=1.6604}{\camel}%
\adviedit{x=4.7780,y=-1.2986,w=-0.6045,h=0.6559}{\camel
\adviedit{x=0.3666,y=0.5704,w=0.6741,h=0.8008}{\bubble
        {}(\advicw,\advich){Joe}}}%
\adviedit{d=3.1698}{\vtop{\vspace\advid}}%
\hspace{\adviw}%
}%
\end{toimage}\imageflush
}
\caption{\label{caravan}A Whizzy-editable Camel Caravan}
\end{figure}
Below is another example with two circles:
\begin{verbatim}
        \adviedit[A]{w=4}
          {\setedit{unit=\adviw}%
           \psset{boxsep=0pt,framesep=0pt}%
           \hbox to \adviw
             {\circlenode{A}{\hspace {\adviw}}\hss
              \adviedit[B]{w=0.5}{\circlenode{B}{\hspace{\adviw}}}}}
\end{verbatim}
Many {\LaTeX} commands such as \texttt{\string\hspace},
\texttt{\string\parbox}, \emph{etc.} are parameterized by dimensions.
However, some other commands, such as \texttt{\string\picture}, 
\texttt{\string\pspicture} and  most PsTricks commands,
\texttt{\string\bubble}, and \texttt{\string\adviedit} itself are
parameterized by a coefficients (floats) and, separately, a dimension. 

To whizzy-edit such coefficients, there are also commands
\texttt{\string\advicx},
\texttt{\string\advicy},
\texttt{\string\advicw},
\texttt{\string\advich}, and 
\texttt{\string\advicd} that contain the float ratio of 
the corresponding dimension with respect to
\texttt{\string\adviunit}---whenever the dimension is itself defined.
As an example, the position of bubble can whizzy-edited as follows:
\begin{verbatim}
        \adviedit{h=1.8902,w=1.5259,unit=\bubbleunit}
          {\bubble{anchored text}(\advicw,\advich){bulle text}}
\end{verbatim}

\subsection{Writing whizzy-editable macros}

Although the command \texttt{\string\whizzyedit} is quite general and
powerful, the user may wish to write its own versions.  
One must then be careful that the macro correctly passes its name to 
{\ActiveDVI}. For instance, rebinding or partially evaluating the macro
\texttt{\string \adviedit} does not work, since then the text-source macro
will not be \texttt{\string \adviedit} anymore. See the latex
\texttt{advi.sty} source package for envolved examples. 

Below are just a couple of simple examples. 
We can abbreviate the example of adjustable horizontal spaces defining the
following macro: 
\begin{verbatim}
        \newcommand{\advihspace}[1]
           {\adviedit{comm=\advihspace,#1}{\hspace{\adviw}}}
\end{verbatim}
The argument \lst"comm=\advihspace" set the name of the calling source text
macro to \lst"\advihspace". Then, you may simply write: 
\begin{verbatim}
        \advihspace{w}
\end{verbatim}
instead of 
\begin{verbatim}
        \adviedit{w}{\hspace{\adviw}}
\end{verbatim}
The macro could additionally check that \lst"w" is indeed defined.

Another example of specialization is to place bubbles: so as to be more
intuitive, the orgin of the edition should start at the center rather then
at the left of the anchor, which requires a small acrobatics with boxes and
dimensions: 
\begin{verbatim}
        \newcommand{\editbubble}[3]
           {\setbox0=\hbox{#2}\copy0\hbox to 0em {\kern-0.5\wd0\relax
            \bbb@dima=\ht0\bbb@dimb=\dp0
            \setbox0=\null\ht0=\bbb@dima\dp0=\bbb@dimb
              {\adviedit{comm=\editbubble,unit=\bubbleunit,#1}
                 {\bubble{\box0}(\advicw,\advich){#3}}}\hfilneg}}
\end{verbatim}
Then a nicely editable bublle can be obtained with
\begin{verbatim}
        \editbubble{w,h}
          {\editbubble{w,h}
            {\editbubble{w,h}{flowers}{First}}
            {Second}}
          {third}
\end{verbatim}
(See the result in Figure~\ref{flowers})
\begin{figure}
$$
\begin{toimage}
\makeatletter
\newcommand{\editbubble}[3]
           {\setbox0=\hbox{#2}\copy0\hbox to 0em {\kern-0.5\wd0\relax
            \bbb@dima=\ht0\bbb@dimb=\dp0
            \setbox0=\null\ht0=\bbb@dima\dp0=\bbb@dimb
              {\adviedit{comm=\editbubble,unit=\bubbleunit,#1}
                 {\bubble{\box0}(\advicw,\advich){#3}}}\hfilneg}}
\makeatother
\vbox{\vspace{2em}
\hbox{\editbubble{w=3.5896,h=0.5549}
{\editbubble{w=0.3121,h=1.1212}
            {\editbubble{w=-2.6822,h=0.5145}{anchor}{first}}
            {second}}
          {third}
}}
\end{toimage}\imageflush
$$
\caption{\label{flowers}A bubble flower}
\end{figure}

%HEVEA\hypertarget{impl}{}
\section{A quick overview of the implementation}

In short, {\sc \whizzy} is selecting a small slice of the document that 
you are editing around the cursor (according to the selected mode) 
and redisplay the slice incrementally as it changes through edition. 
\begin {itemize}

\item {\bf Emacs is watching you} typing and moving in the 
Emacs buffer attached to the {\LaTeX} source file that your editing and keeps
saving the current slice (current slide, section, or subsection, according
to the mode).

\item {\bf A shell-script daemon}
keeps recompiling whenever a new slice (or other files) are produced, and if
recompilation succeeds, tels the previewer to updates the display of the slice.

\item {\bf A few {\LaTeX} macros} allow to build a specialized
format with all macro loaded, which considerably speed up the time for
slicing. Additionally, the slice is a bit instrumented to show the cursor,
and includes specials that allows back-pointing from the DVI file into the
Emacs buffer.

\end {itemize}
The rest of this section briefly describe these three parts\footnote {This
section is not quite up-to-date, hence it puts emphasis on the original
design, but several aspects have changed significantly since the first
version. Implementation of more recent features is thus omitted.}, and
their interactions.


\subsection {Emacs code}

The main trick is to use \lst"post-command-hook" to make Emacs watch
changes.  It uses \lst"buffer-modified-tick" to tell if any editing has
actually occurred, and compare the point position with the (remembered)
position of the region being displayed to see if saving should occur.  It
uses \lst"sit-for" to delay slicing until at least the time of slice
computation has ellapsed since last saving, a significant number of editing
changes has occurred, or iddleness.

{\whizzy} can also display the cursor position, in which case slices are
also recomputed when the cursor moves, but with lower priority.

\subsection {{\LaTeX} code}

The main TeX trick is to build a format specialized to the current
document  so as to avoid reloading the
whole macros at each compilation. This is (almost\footnote{{\tt
$\backslash$begin\{document\}} should be typed as such without any white
white space}) entirely transparent, that is, the source file does not have
to understand this trick.

This is implemented by redefining \lst"\documentclass" which in turn
redefines \lst"\document" to execute \lst"\dump" (after redefining
\lst"\document" to its old value and \lst"\documentclass" so that it skips
everything till \lst"\document"). This is robust ---and seems 
to work with rather complex macros. 

The specialized format can be used in two modes: by default it expects a
full document: it them dumps counters at sectioning commands (chapters, 
sections, and subsections). This is useful to correctly 
numberred sections and pages on slices. 

There are also a a few other used to get more advanced behavior, especially
to dump source line numbers and file names so that the previewer can
transform clicks into source file positions. 

When building the format, {\whizzy} also look for a local file of name
\lst"whizzy.sty", which if existing is loaded at the end of the macros. 
This may be used to add other macros in {whizzy} mode, {\em e.g.} 
some {\TeX} environments may be redefined to changed they type setting,
according to whether the current line is inside or outside the environment. 
(We have written such an extension for an exercise package that sends the
answers at the end in an appendix, unless the cursor is inside the answer,
in which case the answer is in-lined.)

\subsection {Bash code}

There is no real trick there. This is a shell-script watching the pool
(a directory where slices and other new version of files must be dropped). 
It them recompiles a slice and wait for input (in stdin). 
It recognizes a few one-line commands as input {\tt reformat}, {\tt
dupplex}, and by default just watch for the presence of a new slice. 
It recompiles the format file (and the page and section number, but in batch
mode) whenever the source file (its Unix date) has changed  and 
recompiles the slice whenever it is present (since {\whizzy} renames ---hence
removes--- the slice before processing it).

If the file has been recompiled successfully, it triggers the previewer
(ghostscript or xdvi) so that it rereads the dvi or ps file. Otherwise, it
processes the {\TeX} log file and tries to locate the error. It then sends part
of the log file with annotations to the \lst"*TeX-shell*" buffer from which
Emacs has been {\whizzy}, so that Emacs can report the error. 

\subsection {Interaction between the components} 

The control is normally done by Emacs, which launches and kills the Unix
daemon. Quitting the previewer should be noticed by the daemon, which tells
Emacs to turn mode off before exiting. 

Muliple {\whizzy} running on the same file would certainly raise racing
conditions between files and would not make sense. 
For that purpose, the daemon pid is saved in a file and {\whizzy}
will kill any old {\whizzy} process on startup. 

\subsection {Whizzy edition}

The macros \texttt{\string\adviedit} passes information
to {\ActiveDVI} inside \texttt{edit} specials. This information is used to
identify the source file command that requested some edition and is passed
by from {\ActiveDVI} to emacs as command strings of the form: 
\begin{quote}
\begin{verbatim}
<edit "\adviedit" ""[x=1.2001]" #56 @main.tex moveto 5.1529,-1.1708>
\end{verbatim}
\end{quote}
This command emitted by {\ActiveDVI} in its standard output is thus received by
emacs via {\Whizzy} in the process buffer associated to the current
session. 

Emacs interprets such commands starting with the ``\verb"<edit "'' prefix 
as whizzy edition commands. In the above example, the string
\verb"\adviedit" is a latex commands that should be present the master
buffer \verb"main.tex" at line \verb"56" and with x coordinate equal to
\texttt{1.2001}. Its \texttt x and \texttt y coordinates should be 
changed by \texttt 5.1529 and \texttt{-1.1708}.  Usually, the command can be
precisely located by its line position in the buffer and one significant
coordinates.  In case of conflict, a tag optional argument pass
\texttt{\string\adviedit} will be passed to {\ActiveDVI} and then sent back
to emacs (which is filled in the empty string above).


\end{document}

- Clicks

- Say that cursor does not always appear.
% LocalWords:  whizzy advi html whizzytex HeVeA BGCOLOR Didier INRIA ee ou ing
% LocalWords:  ghostview xdvi DVI XEmacs Redhat tgz untar gunzip xf usr config
% LocalWords:  tex minibuffer FMT online dvi dview args autoload ps pre mv
% LocalWords:  BASENAME hyperref gv HUP pid dvips xvf cd wget WhizzyTeX WYSIWIG
% LocalWords:  RET preprocess basename fmt hugelatex BASICFILENAME SIGHUP recog
% LocalWords:  SIGUSR nizes acroread localmacros ghostscript
