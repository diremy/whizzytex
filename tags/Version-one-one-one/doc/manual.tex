\documentclass[12pt]{article}

\usepackage {fullpage}

\usepackage {color}
\usepackage {hyperref}
\input {version}

\let \lst \verb
%HEVEA \def \space { }\let \hfil \relax \let \hfill\relax

\def \whizzy {{Whizzy\kern -0.3ex\raise 0.2ex\hbox{\let \@\relax\TeX}}}
\def \Whizzy{\textbf {\textcolor {blue}{\whizzy}}}
\def \instruction #1{\par\medskip \noindent$\Rightarrow$ {\em #1}}
\def \WhizzyEdit {Whizzy{\sc 
\raise 0.2ex \hbox{E}\kern -0.2ex%
\lower 0.0ex \hbox{d}\kern -0.2ex%
\lower 0.2ex \hbox{i}\kern -0.5ex%
\raise 0.2ex \hbox{T}}}%


%% The following lines are to help HeVeA make  the HTML version of the manual

%HEVEA \def \instruction #1{}
%HEVEA\def \whizzy{{Whizzy{\TeX}}}
%HEVEA\def \WhizzyEdit {{WhizzyEdit}}
%HEVEA\renewcommand{\@bodyargs}{TEXT=black BGCOLOR=white}

\usepackage {manual}

\begin{document}
\pagestyle {empty}
\author {\rightline {Didier R{\'e}my}}
\date {\hfill Version {\version}, \today}
\title {
\hfilneg{\huge \Whizzy\footnote{Whizzytex is free software, 
Copyright \copyright 2001, 2002 INRIA
and distributed under the GNU General Public License
(See the COPYING file enclosed with the distribution).}}
\\[1em]
{\em
%HEVEA An {\bfseries Emacs} minor-mode \\
%HEVEA for {\bfseries incremental viewing of} \\ 
%HEVEA {\bfseries {\LaTeX} documents}
%BEGIN LATEX
\vbox {%
\centerline
{{\raggedright \normalsize \activedvi {Whizzy-}{Te\/X}{WhizzyTeX}}
\hfill
{\raggedright \normalsize \emacs {Whizzy-}{Te\/X}{WhizzyTeX}}
\hfill
\vtop {\vskip -5em \hbox {%
\begin{tabular}{r@{}}
An {\bfseries Emacs} minor-mode for\\
{\bfseries incremental viewing of} \\ 
{\bfseries {\LaTeX} documents}
\end{tabular}}}}\vskip -1em}
%END LATEX
}
}

\maketitle   


\begin{abstract}
\def \B{\textbf}
{\whizzy}
% \footnote {Standing for {\em {\B W}hat {\B i} {\B z}ee {\B i}z what
% {\B y}ou {\expandafter \B \TeX}}}
is an Emacs minor mode for incrementally
viewing {\LaTeX} documents that you are editing.
%
It works under Unix with {\tt gv} and {\tt xdvi} viewers, but 
the \href{http://pauillac.inria.fr/advi/}{Active-DVI} viewer will
provide much better visual effects and offer more functionalities.
\end{abstract}


\section {Installation}

To use whizzytex, you need {\tt Emacs} or {\tt XEmacs}, {\tt latex2e}, and
{\tt bash} installed, and at least one DVI or Postscript previewer, such as
{\tt advi}, {\tt xdvi}, or {\tt dvips} combined with {\tt gv}. 

{\whizzy} has been developed under Linux Redhat 7.2 but has not been
extensively tested on other platforms. However, {\LaTeX} and Emacs are quite
portable and possible compatibility problem with the bash shell-script
should be minor and easily fixable.

Get the source {\tt whizzytex-\version.tgz} 
from the \href{http://pauillac.inria.fr/whizzytex}{distribution}, 
uncompress and untar it in some working directory, as follows:
\begin{quote}
\begin{tt}
gunzip whizzytex-\version.tgz\\
tar -xvf whizztex-\version.tar\\
cd whizzytex-\version
\end{tt}
\end{quote}
Then, the installation can be automatic (default or customized), or manual.

\subsection {Automatic installation}

\label {install/automatic}

By default, shell-script \lst"whizzytex" will be installed in 
\lst"/usr/local/bin/" and all files will be installed a subdirectory
of \lst"/usr/local/share/whizzytex/" 
but the documentation, which will be installed in
\lst"/usr/local/share/doc/whizzytex/". 
Moreover,  Emacs-lisp code will not be byte-compiled. 

For default installation, just type: 
\begin{quote}
\begin{tt}
make all
\end{tt}
\end{quote}
This will create a \lst"Makefile.config" file (only if nonexistent) by
taking a copy of the template \lst"Makefile.config.in". This will also check
the \lst"Makefile.config" (whether it is the default or a modified version)
against the software installed on your machine.  If you wish to change the
default configuration, or if your configuration is rejected, see ``{\bf
Customizing the installation}'' below. This will also prepared configured
versions of the files for installation.

Finally, to install files, become superuser (unless you are making 
an installation for yourself) and do:
\begin{quote}
\begin{tt}
make install
\end{tt}
\end{quote}
The first line ensures that you give read and execute permission to all.

See {\bf Using {\whizzy}} (Section \ref {using}) to test your
installation.

\subsection {Customizing the installation}

To customize the installation, you can edit 
\lst"Makefile.config", manually.
You may also use either the command
\begin{quote}
\begin{tt}
./configure
\end{tt}
\end{quote}
This command may be passed arguments to customize your installation.
Call it with the option \lst"-help" to see a list of all options.
%
By default, the configuration is not interactive.  However, you may call it
with option \lst"-helpme" to have the script do more guessing for you and
prompt for choices if needed.

Note that by default, the Emacs-lisp code whizzytex.el is not
byte-compiled. You need to pass the option \lst"-elc" to \lst"configure" in
order to byte-compiled it.

\paragraph {Checking {\tt Makefile.config}}

A misconfiguration of your installation, or ---much more subttle--- a
misconfiguration of other commands (it appears that some installations wrap
scripts around standard commands that are sometimes incorrect and break
their normal advertized interface) may lead to systematic errors when
launching {\whizzy}. To prevent delaying such obvious errors, some sanity
checks are done after \lst"Makefile.config" has been produced and before
building other files.  These include checking for mandatory bindings (useful
for manual configuration) and for the conformance of {\tt initex}, {\tt
latex}, and viewers commands to their expected interface.

Checking viewers interface implies simulating a small {\whizzy} session: a
small test file is created for which a specializled version of latex format
is built and used to run {\LaTeX} on the test file; finally, required
viewers are tested on the DVI output, which opens windows, temporarily.

If the sanity check fails, at least part of your configuration is 
suspicious. If some windows remain opened, your confirguration is likely to
be erronesous (and so, even if not detected by the script).  

However, if you really know what you are doing, you may bypass the check by
typing \lst"make config.force", which will stamp your \lst"Makefile.config"
as correct without checking it. Checking compliance to viewers interface
is also bypassed if you you do not have a connection to X. Conversely, you
may force checking manually by typing \lst"./checkconfig".

At the end of customization, proceed as described in ``{\bf Automatic
installation}'' (Section \ref {install/automatic})

\paragraph {Customization notes}

By default, {\whizzy} assumes the standard convention that 
\lst"latex" is the command name used to call {\LaTeX}, 
\lst"initex", the command name used to build a new format, and that the latex 
predefined format is \lst"latex".

If your implementation of {\LaTeX}
uses other names, you may redefine the variables \lst"INITEX",
\lst"LATEX", and \lst"LATEXFMT" accordingly in the file
\lst"Makefile.config".
%
For instance, \lst"platex" could be use the default configuration
\begin{quote}
\begin{tt}
INITEX = iniptex\\
LATEX = platex\\
LATEXFMT = platex\\
BIBTEX = jbibtex
\end{tt}
\end{quote}
This would be produced directly with the configuration line:
\begin{quote}
\begin{tt}
./configure -initex iniptex -latex platex -latexfmt latex -bibtex jbibtex
\end{tt}
\end{quote}
If you wish to run {\whizzy} with several configurations, you must still
choose a default configuration, but you will still be able to call {\whizzy}
with another configuration from Emacs (see Section \ref{configuration.tex}
below).

It is possible to customize the set up on a per-user basis by creating
a file \lst"~/.whizzytexrc" containing, for example, the following lines: 
\begin{quote}
\begin{tt}
INITEX = iniptex\\
LATEX = platex\\
FORMAT = platex\\
BIBTEX = jbibtex
\end{tt}
\end{quote}

During the configuration, you must at least choose one default previewer
type among \lst"advi", \lst"xdvi", and \lst"ps", and at most one default
previewer for each previewer type you chose. You will still be able to call
{\whizzy} with other previewers from Emacs, via Emacs configuration (see
Section
\ref {configuration.viewers}). 

\subsection {Manual installation}

Since {\whizzy} only need three files to run, installation can also be done
manually:
\begin {itemize}

\item[]\hspace{-2em}{\tt whizzytex.el}

This could be installed in a directory visible by Emacs, but does not need
to, since you can always use the full path when you load it or declare
autoload. 

No default location.

\item[]\hspace {-2em}{\tt whizzytex}

This file is a bash-shell script that should be executable.  There is not
reason to have it visible from the executable path, since it should not be
used but with {\whizzy}.

The variable {\tt whizzytex-command-name} defined in {\tt whizzytex.el} 
contains its full path (or just its name if visible from the executable
path). 

Default value is \lst"/usr/local/bin/whizzytex"

You may  need to adjust the path of \lst"bash" in the very first line of the
script, as well as some variables in the manual configuration section of the
script. 

\item[]\hspace{-2em}{\tt whizzytex.sty}

This file are {\tt latex2e} macros. There is no reason to put this visible
from {\LaTeX} path, since it should not be used but with {\whizzy}.

Variable variable {\tt PACKAGE} defined in {\tt whizzytex} 
the full path (or just the name if the path is visible from {\LaTeX}. 

Default value is \lst"/usr/local/share/whizzytex/latex/whizzytex.sty"

\end {itemize}


\subsection {Automatic upgrading}

For convenience, the distribution also offers a facility to download and
upgrade new versions of {\whizzy} (this requires \lst"wget" to be
installed).  If automatic upgrading does not work, just do it manually.

All operations should be performed in the {\whizzy} top directory, {\em
i.e.} where you untar whizzytex for the first time, that is right above the
directory from were you made the installation. We assume that have 
created a link to the current version subdirectory: 
\begin{quote}
\begin{tt}
ln -s whizzytex-\version\space whizzytex
\end{tt}
\end{quote}
(the manager will then update this link when version changes).
Alternatively, you can also use the full name {\tt whizzytex-\version} in
place of {\tt whizzytex} below. The main commands are:
\begin{quote}
\begin{tt}
make -f whizzytex/Manager upgrade \\
make -f whizzytex/Manager install
\end{tt}
\end{quote}
The command \lst"upgrade" will successively download the newest version,
unpack it, copy the configuration of the current version to the newest
version, and bring the newest version up-to-date. The command \lst"install"
will install files of the newest version. 

The following command will (re-)install an old version:
\begin{quote}
\begin{tt}
make VERSION=<version> download downgrade install
\end{tt}
\end{quote}

\section{Using {\whizzy}}
\label{using}

\subsection {Loading {\tt whizzytex.el}}

Maybe, {\tt whizzytex} is already installed on your (X)Emacs system, which
you may check by typing:
\begin{quote}
\begin{tt}
ESC x whizzytex-mode RET
\end{tt}
\end{quote}
If the command is understood, skip this section.
Otherwise, you should first load the library \lst"whizzytex.el" or, better,
declare it autoload. To do this permanently, include the following
declaration in your Emacs startup file (probably is \lst"~/.emacs").
\begin{quote}\small
\begin{tt}
(autoload 'whizzytex-mode \\ \indent\obeyspaces
    "/usr/local/share/whizzytex/lisp/whizzytex.el" \\ \indent\obeyspaces
    "WhizzyTeX, a minor-mode WYSIWIG environment for LaTeX" t)
\end{tt}
\end{quote}
(where \lst"/usr/local/share/whizzytex/lisp/whizzytex.el" is the exact
location of \lst"whizzytex.el", which depends on your installation: type
{\tt make where} to see where Emacs Lisp was installed.)  If
\lst"whizzytex.el" happens to be in your (X)Emacs {\tt load-path}, or if you
have adjusted this variable appropriately, you can simply write:
\begin{quote}\small
\begin{tt}
(autoload 'whizzytex-mode "whizzytex" \\ \indent \obeyspaces
    "WhizzyTeX, a minor-mode WYSIWIG environment for LaTeX" t)
\end{tt}
\end{quote}

\subsection {Quick start} 

{\whizzy} runs as a minor mode of Emacs to be launched on a {\LaTeX} Emacs
buffer. The extension of the buffer should be
\lst".tex".  {\whizzy} also understands \lst".ltx" extensions, but gives
priority to the former when it has to guess the extension. Other extensions
are possible but not recommended.
\begin{quote}\em
The file attached to the buffer must exists and either be a well-formed
{\LaTeX} source file, or be {\em mastered}, {\em i.e.} loaded by another
{\LaTeX} source file. Thus, whenever the buffer does not contain a
\lst"\begin{document}" command), {\whizzy} will search for its master file,
asking the user if need be, so as to first launch itself on a buffer
visiting the master file. In particular, an empty buffer will be considered
as beeing mastered, which may not be what you intend.
\end{quote}
To start {\whizzy} on either kind of buffer, type:
\begin{quote}
\begin{tt}
ESC x whizzytex-mode RET
\end{tt}
\end{quote}
By default, this should add new bindings so that you can later turn mode
on and off with key strokes {\tt C-c C-w}. This will also add a new menu
{\tt Whizzy} in the menu bar call ``the'' menu below. (If you are using 
the {\tt auctex}, your may use other configuration key strokes to avoid
clashes (see online emacs-help). 

When {\tt whizzytex-mode} is started for the first time on a new buffer, it
attempts to configure buffer local variables automatically by examining
the content of file, and using default values of global bindings.

You may customize default settings globally by running appropriate
hooks or locally by inserting appropriate comments in the source file ---see
the manual below. 

You may also change the settings interactively using the menu, or tell
whizzytex-mode to prompt the user for confirmation of file configuration by
passing prefix argument 4 (using, for instance, key sequence 
\lst"C-u C-c C-w"). 


\subsection {Editing}

Once {\tt whizzytex-mode} is on, just type in as usual.  {\whizzy} should work
transparently, refreshing the presentation as you type or move into your
{\LaTeX} buffer. 

Additionally, a gray overlay is put outside of the current slice (the {\em
slice} is the region of your buffer under focus, which is automatically
determined by {\whizzy}). When a {\LaTeX} error occurs and it can be
localized in the source buffer, a yellow overlay also is put on the region
around the error, and removed when the error is fixed.

Furthermore, when an error is persistent for several slices or some amount
of time, the interaction-buffer will pop up with the error log
(this option can be toggled on and off with the {\tt Auto interaction} menu
entry).  

\subsection {Recovering from errors}

{\whizzy} makes a good attempt at doing everything automatically. 
However, there remains situations where the user need to understand 
{\whizzy} ---when {\whizzy} does not seem to understand the user anymore. 

For that purpose, {\whizzy} report processing and error messages
in its interaction window. Thus the first help for debugging
is always to look at interaction window (buffer 
\lst"*filename.tex*" (where \lst"filename" stands for the name of the file
associated with the main buffer in case several files are composing your
document). 

This window will pop up and down automatically when an error persists or
disappear. For debugging, you may unset {\tt Auto interaction} so as to see
the interaction buffer permanently. You may also unset {\tt Auto Shrink
output} to keep all log.

The {\tt View Log...} menu entry can be used to view log files of
last actions performed by whizzytex. 

\subsection {Error during initialization}

The most delicate part of {\whizzy} is when starting {\tt whizzytex-mode},
and usually for the first time in a new buffer, since at that time all kinds
of initialization errors may occur (in addition to {\LaTeX} errors. 

{\whizzy} will attempt to identify the error and report appropriate messages
in the interaction buffer. (In case an error occurs ---or nothing happens---
always have a look at the interaction buffer, even if it did not prompted
automatically.)

Here are a description of errors during initialization mostly in
chronological order. 

\paragraph {Emacs fails during setup}

This is all under Emacs, so easily under control.
Normally, Emacs should report error messages. See the documentation for
explanations. 

In case, of uncaught fatal errors, you may
\verb"ESC X toggle-debug-on-error" to get help from Emacs, and try to fix
the problem. 

\paragraph {Emacs cannot find whizzytex}

This should typically be an installation problem, where the variable
\lst"whizzytex-command-name" is erroneous (maybe you need to give the full
path). Try to evaluate \verb"(shell-command whizzy-command-name)" in the
minibuffer, which of course should fail, but only after the command has been
reached.

\paragraph {Whizzytex cannot build a format}

Then {\whizzy} will refuse to start. 

The problem could result from an abnormal interaction between your macros
and {\whizzy} macros, but this situation seems rather unfrequent.  So there
is most probably an error in your macros.  Try to compile {\LaTeX} your
file.  

% You may also try to erase all auxiliary files, since if those are
% erroneous, they may disturb the creation of the format (which loads some of
% the auxiliary files). 


By default the interaction window will pop-up with an section of the format
log. If this is not enough, you may need to view log files.  However, log
files are normally removed when {\whizzy} exits.  To keep log files on,
you must retart {\whizzy} in debug mode (select the debug mode in the
menu and restart {\whizzy}). Then, you can check the \lst"format" log and
if necessary the \lst"command" with which {\whizzy} has been launched.
(Once the bug is fixed, you should switch off the debug mode, which may slow
down {\whizzy}.)


\paragraph {Whizzytex cannot launch the previewer}

Usually, this is because whizzytex received wrong previewer parameter.  See
the command echoed in the interaction buffer or try to evaluate
\lst"(whizzy-get whizzytex-view-mode)".

\paragraph {Other errors}

There are two remaining problems that could happen at launch time, but that
are not particular to launch time: {\whizzy} could not recompiled the whole 
document or the current slice. However, these should not be fatal. 
In the former case, whizzytex will proceed, ignoring the whole document 
(or using the slice instead if you are in duplex mode). In the later case,
whizzytex will replace the slice by an empty slice ---and print a welcoming
document, as if you launch {\whizzy} outside of any slice. 

\subsection {Errors during normal edition}

After initialization time, {\whizzy} will keep recompiles slices as you
type or move, but also recompiles the format and the whole document when you
save a file. Each of this step may failed, but this should not be fatal, and
Emacs should report the error, possible pop up the interaction window, and
continue. 

\paragraph {Whizzytex fails on the current slice}

This should not be considered as an error, it {\bf must} happen during
edition. In particular, {\whizzy} is not much aware of {\LaTeX} and could
very well slice in the middle of the typesetting of an environment or a
macro command. This should not matter, since the erroneous slice will be
ignore temporarily until the slice is correct again.

\paragraph {Whizzytex keeps failing on the current slice}

The slice can also be erroneous because the Emacs did not correctly inferred
where to insert the cursor, which may slice erroneous, although what you
typed is correct. Hopefully, this will not occur too often, and disappear as
you move the point. It should also disappear if you switch off both {\tt
Point visible} and {\tt Page to Point} options, which is actually a good
thing to do when you suspect some misbehavior.  This will make WhizzyTeX
more robust, but less powerful and more boring.

\paragraph {Whizzytex does not seem to slice at all}

The interaction window does not produce any output. 
Try to move in the slice, or to another slice. 

If nothing happens, check the interaction
window, to see if it did attempt to recompile the slice.
If nothing happens in the interaction window, check for Emacs messages
(in the \lst"*Messages*" buffer). You may also check for the presence 
(and content) of the slice by visiting 
\lst"_whizzy_filename.tex" or
\begin{quote}
\begin{verbatim}
_whizzy_filename/input/_whizzy_name.new
\end{verbatim}
\end{quote}
If neither file exists, it means that Emacs did
not succeed to slice, which you may force by evaluating
\lst"(whizzy-observe-changes t)". 
This can be run in even if {\tt whizzytex-mode} is suspended, which may
avoid automatic processing of slices, and their erasure.

If the slice is present, you may try to compile it by hand (outside of
Emacs) with 
\begin{quote}
\begin{verbatim}
latex '&_whizzy_filename' _whizzy_filename.tex
\end{verbatim}
\end{quote}
and see if it succeeds. 


\paragraph {Reformatting failed}

Formatting errors are fatal during initialization, but accepted once
initialized. In case of an error during reformatting, {\whizzy} will ignore
the error and continue with the old format.  This means that new macros may
be ignored leading to further slicing errors. When rebuilding the format
failed, the mode-line string will display the suffix \lst"FMT" until the
error is fixed.  See the interaction buffer or select \lst"format" from the
\lst"log..." menu entry).

You may also force reformatting by typing the \lst"reformat" command
in the interaction buffer. 


\paragraph {Whizzytex cannot process the whole document}

This is very likely a problem with you document, so try to {\LaTeX} it 
first. There is a small possibility of strange interaction between
your macros and {\whizzy} package. Try to turn options 
{\tt Page to Point} and {\tt Point visible} off and retry. 
This will make {\whizzy} more robust (but also less powerful and more
boring). 

\subsection {Debugging}

If you are completely lost, and none of the above helped, you may need some
debugging.  If this is your first attempt at {\whizzy}, you should suspect your
configuration. You should them try with the examples of the distribution. 
Otherwise, you may rollback to a file and configuration that used to work, 
and make incremental changes.

You can turn emacs debug mode on and off with
\begin{quote}
\begin{verbatim}
ESC x toggle-debug-on-error RET
\end{verbatim}
\end{quote}
You can also turn whizzytex debug mode on and off by typing either line
in the interaction window:
\begin{quote}
\begin{verbatim}
trace on
trace off
\end{verbatim}
\end{quote}
If whizzytex fails at lauch time, you may also use the option {\tt -trace}
of file configuration (see Section~\ref {configuration.trace}).


\section{Manual} 
\label{manual}

This section describes how to use and parameterize {\whizzy}.  So as to
avoid redundancy, {\bf most of the documentation is only available online in
Emacs, from the \lst"Help" entry of the menu by following hyperlinks.}
%
Alternatively, you can type
\begin{quote}
\begin{verbatim}
ESC x describe-function RET whizzytex-mode RET
\end{verbatim}
\end{quote}
(In XEmacs, you may need to use
\begin{quote}
\begin{verbatim}
ESC x hyper-describe-function RET whizzytex-mode RET
\end{verbatim}
\end{quote}
instead of \lst"describe-function" to see hyper-links.)

Section \ref{configuration}, \ref{modes} and
\ref{types} are also available as online help. 

\subsection {Configuration} 

\label{configuration}

\subsubsection {Emacs global configuration}

\label{configuration.viewers}
\label{configuration.bindings}

\label {Emacs-configuration}

See Emacs help for \lst"whizzy-default-bindings" and
\lst"whizzytex-mode-hook" for list of bindings.

The Emacs on-line help for \lst"whizzytex-mode" lists all user-configurable
variables,  which  may be given default values in your Emacs startup file
to be used instead of {\whizzy} own default values. 

\subsubsection {File-based configuration}

\label {File-configuration}


A configuration line is one that starts with regexp prefix ``\lst"^%; +"''
followed by a configuration keyword.  If two configuration lines have the same
keyword, only the first one is considered. The argument of a configuration
line is the rest of the line stripped of its white space.

The keywords are:
\def \arg#1{$\langle\texttt {#1}\rangle$}
\def \opt#1{[ #1 ]}
\begin{description}
\item[whizzy-master]\arg {master}
\\
This only makes sense for a file loaded by a {\em master} file. 
\arg{master} is the relative or full name of the
master file. Optional surrounding quotes (character \lst`"`) %" 
%
stripped off, so that \lst$"foo.tex"$ and \lst"foo.tex" are equivalent.

\item[whizzy] 
\opt{\arg{slicing}} 
\opt{\arg{viewer} \opt{\arg{command}}}\\
\opt{\texttt{-mkslice} \arg{command}} 
\opt{\texttt{-mkfile} \arg{command}} \\
\opt{\texttt{-tex} \arg{suffix}} 
\opt{\texttt{-initex} \arg{initex}}
\opt{\texttt{-latex} \arg{latex}}
\opt{\texttt{-fmt} \arg{format}}\\
\opt{\texttt{-bibtex} \arg{bibtex}}
\opt{\texttt{-dvicopy} \arg{command}}
\opt{\texttt{-watch}}
\opt{\texttt{-duplex}}
\opt{\texttt{-trace}}
\\[1em]
All arguments are optional, but if present they must appear in order and on
a unique line:
\begin{description}
\item[\arg{slicing}]\indent\\ 
is determines the way the document is sliced
(see section~\ref{modes}).

\item[\arg{viewer}]\indent\\
is the type of viewer and can only be one of 
\lst"-advi", \lst"-xdvi", or \lst"-ps" (see section~\ref{types})

\item[\arg{command}]\indent\\
is optional and is the command used to call the viewer
(of course, it should agree with \arg{viewer}). 

\item[\texttt{-mkslice} \arg{command}]\indent\\
tells {\whizzy} to use  \arg{command} to preprocess the slice. 
The command \arg{make} will receive one argument
\texttt{\_whizzy\_basename.new} and should produce 
\texttt{\_whizzy\_basename.tex}
(or \texttt{\_whizzy\_basename.ltx} if the extension of the master file is 
\texttt{.ltx}).
By default, \lst"mv" is simply used.

{\em The Unix \lst"make"  can itself be used as a preprocessor (with an
appropriate \lst"Makefile").  However, one may have to work around
\lst"make"'s notion of time (using FORCE), which is usually too rough. 
This is safe, since {\whizzy} tests itself for needed recompilations.}

\item[\texttt{-mkfile} \arg{command}]\indent\\
executes ``\arg{command} \arg{filename}'' before recompiling every time a
buffer is saved. The argument ``\arg{filename}'' is the buffer-file-name
path relative to the path of the master file directory.

\item[\texttt{-makeindex} \arg{command}]\indent\\
uses ``\arg{command} \arg{filename.idx}'' for rebuilding the index instead
the default ``\arg{makeindex} \arg{filename.idx}''.  If ``\arg{command}'' is
false, then do not attempt to rebuild the index.

\item[{\bf {\tt -bibtex \arg{bibtex}}}]\indent

uses \arg{bibtex} for the bibtex command instead of the value 
assign to BIBTEX in \lst"Makefile.config" (or \lst"whizzytex")

\item[{\bf {\tt -initex \arg{initex}}}]\indent

uses \arg{initex} for the initex command instead of the value 
assign to INITEX in \lst"Makefile.config" (or \lst"whizzytex")

\item[{\bf {\tt -latex \arg{latex}}}]\indent

uses \arg{latex} for the latex command instead of the value 
assign to LATEX in \lst"Makefile.config" (or \lst"whizzytex")

\item[{\bf {\tt -fmt \arg{format}}}]\indent

uses \arg{format} for the latex format instead
of the default value, usually fmt (see configuration).

{\em This can either be used in combination with \lst"-latex" and
\lst"-initex", 
or alone. For instance,
\lst"hugelatex" could be used (depending on your {\LaTeX} configuration) to
build a larger format to process huge files.}
\label{configuration.tex}

\item[{\bf {\tt -dvicopy \arg{command}}}]\indent
\label {sec/dvicopy}

uses \arg{command} instead of the default (mv) to copy DVI files
(from \lst"FILE.dvi" to \lst"FILE.wdvi"). This can be used with command
\lst"dvicopy" so as  to expand virtual font, which advi does not understand
yet) 

\item[\texttt{-watch}]\indent\\
watches other files than just the slice (see Section~\ref {sec/watch}).

\item[\texttt{-duplex}]\indent\\
launches another window with the whole document (which is
recompiled every time the source buffer is saved).

{\em With \lst"-advi" previewers, both views  communicate with Emacs and can be
used to navigate through source buffers and positions.}

\item[\texttt{-trace}]\indent\\
traces all script commands (for debugging purposes only.)
\label{configuration.trace}

\end{description}

For instance, a typical configuration line will be:
\begin{verbatim}
   %; whizzy subsection -dvi "xdvi -s 3"
\end{verbatim}
It tells whizzytex to run in subsection slicing mode and use a \lst"dvi"
style viewer called with the command
\lst"xdvi -s 3". This is also equivalent to
\begin{verbatim}
   %; whizzy subsection -dvi xdvi -s 3
\end{verbatim}
since Emacs removes outer double-quotes in option arguments. 

A more evolved configuration line is:
\begin{verbatim}
   %; whizzy -mkslice make -initex iniptex -latex platex -fmt platex
\end{verbatim}
It tells {\whizzy} to use \lst"iniptex" and \lst"platex" comands instead
of \lst"initex" and \lst"latex" and to use the format file \lst"platex.fmt" 
instead of \lst"latex.fmt". Moreover, it should use \lst"make" to preprocess
the slice.

\item[whizzy-paragraph] \texttt{regexp}\\
This sets the Emacs variable \lst"whizzy-paragraph" to \texttt{regexp}.
\end{description} 


\subsection {Modes} 
\label {modes}

{\whizzy} recognizes three modes \lst"slide", \lst"section", and \lst"document". 
The mode determines the slice of the document being displayed and how
frequently updates occurs. 
\begin{description}

\item [slide]

The mode \lst"slide"  is mainly used for documents of the class seminar. 
In slide mode, the slide is the text between two \lst"\begin {slide}"
comments (thus,  the text between two slides is displayed after the
preceding slide).  

In slice modes, overlays are ignored {\em i.e.} all overlays all displayed in
the same slide, unless a command
\lst"\overlay {"$n$\lst"}" occurs on the left of the point, on the same line
(if several commands are on the same line, the 
right-most one is taken), in which case only layers $p \le n$ are displayed.

\item [section]
In \lst"section" mode, the slice of text is the current chapter, section.

\item [subsection]
As \lst"section" but also slice at subsections. 

\item [paragraph]
The \lst"paragraph" mode is a variation on section mode where, the separator
\lst"whizzy-paragraph" is defined by the user (set to two empty lines by
default) instead of using \lst"\section"  and \lst"\subsection" commands. 
subsection.

\item [document]
The \lst"document" take the region between \lst"\begin{document}"
and \lst"\end"\lst"{document}" as the slice. 

\item [none]
In \lst"none" slicing mode, there is no sectioning unit at all and
the whole document is recompiled altogether. 
Currently, pages are not turned to point and the 
cursor is not shown in \lst"document" mode, because full documents are not
sliced. (A slicing document mode could be obtained by working in paragraph
mode, with an appropriate regexp.)

\end{description}

\subsection {Viewer types}
\label {types}

See help for \lst"whizzy-viewers".

The previewer types can have three possible values:
{\tt -ps}, {\tt -dvi}, or {\tt -advi}. 

The previewer type should agree with the previewer command in several ways:
\begin {itemize}

\item
They tell whizzytex whether to use signal {\tt SIGHUP} (with {\tt -ps})
or {\tt SIGUSR1}  ({\tt -dvi} and {\tt -advi}) to tell the previewer to
reload the file

In particular, if you write a front-hand shell-script \lst"viewer" to the
call previewer,  and want to use \lst"viewer" as the previewer, you should 
arrange for \lst"viewer" to understand these signals (and forward them to the
previewer). The simplest way is to hand your script with an exec command
calling the \lst"gv", \lst"dvi" or \lst"advi".



\item
They tell whizzytex whether to process the slice to 
Postscript (with {\tt -ps}) or to DVI format (with {\tt -dvi} and {\tt -advi}). 

\item
Moreover, {\tt -advi} requires the previewer to 
recognize additional \lst"\special" commands, in particular
source line information of the form: 
\begin{quote}
\begin{verbatim}
#line 780, 785 <<to<<rec>><<ognize>>additional>> manual.tex
\end{verbatim}
\end{quote}

\end {itemize}
Then, the previewer command is the command to call the previewer.  This
string will be passed as such to the {\whizzy} shell-script. Note that the
name of the \lst"dvi" or postscript file will be appended to the previewer
command.

\subsection {Watching other files}
\label {sec/watch}

{\whizzy} is designed to watch other files and not just the slice saved by
Emacs. In fact, it watches any file dropped in the pool directory. 
For instance, 
if your source file uses images, you can just change the image and
drop the new version in the pool. Then {\whizzy} will pick the new version,
move it to the working directory and recompile a new slice. Be aware of name
clashes: if you drop a file in the pool, it will automatically be move to
the working directory with the same name, overriding any file of the same
name sitting there. 

However, activity is entirely controlled by Emacs, since after every
iteration {\whizzy} waits for Emacs to send a new command (usually the empty
command that means iterate again). Hence, other files will only be taken
into account at the next iteration. If you really wish these files
to be watched you need to instrument emacs to send and empty line input to
the interaction buffer regularly, even when idle. 

\subsection {Frequency of recompilation} 

To obtain maximum {\whizzy} effect, a new slice should be save after any
edition changed or any displacement that outside of the current slice.
However, to avoid overloading the machine with useless and annoying
refreshments, some compromise is made, depending on Emacs several
parameters: edition {\em v.s.} move Emacs last commands, 
successful {\em v.s.} erroneous last slice, and the duration of last slice
recompilation. This usually works well. However, different behavior may wish
to be obtained in different situations. For instance, when editing on a
lab-top, one may wish to save batteries by keeping the load rather low, hence
not using the full power of the processor. Conversely, one may wish 
{\whizzy} to be as responsive as possible. There is an function
\lst"whizzy-load-factor" that control a variable of the same name, which can
be used to adjust the responsiveness (by increasing or decreasing the
load-factor). This simply adds extra delays between slicing. 

The format is automatically recompiled at the beginning of each session, and
whenever the buffer containing the file is saved. That is, to load new
packages or define new global macros (before the \lst"\begin{document}"), it
suffices to save the current file.

\subsection {\whizzy-ing macro files} 

Macro files (with \lst".sty" extension) can be whizzytex as well.  The
effect is them only to automatically call \lst"reformat" when the file is
saved.


\subsection {Cross-references, page and section numbers} 

The slice is always recompiled with the \lst".aux" file of the whole
document.  In paragraph mode, cross references and section numbers are 
recompiled whenever the buffer itself is saved (manually). 
The recompilation of the whole document is off in slide mode.


\section{Viewers}

\def \ActiveDVI {Active-DVI}

\subsection {Viewing with \ActiveDVI}

\href{http://pauillac.inria.fr/advi/}{\ActiveDVI} is a DVI previewer with
several additional features. 
In particular, it recognizes extra specials, some of which are particular 
useful for whizzytex that allows a two way communication between 
the source Emacs buffer and the previewer: 
\begin {itemize}
\item
The previewer will automatically turn pages for you, as you are editing. 
This is done by telling Emacs to save the current position in the slice. 
Then, the recompilation of the slice will include the current position 
as an hyperref location \lst"Start-Document" whenever possible. 
Then, just tell {\ActiveDVI} to automatically jump at this location
when it opens/reloads the file. 

\item
Conversely, {\ActiveDVI} can dump source file positions on clicks, 
when available (usually on \lst"shift-left-mouse" or \lst"left-mouse"), that
is forwarded to Emacs so that it can move to the corresponding line.

To enjoy this feature, the option \lst"-advi" should be used instead of
\lst"-dvi". This will produce extra information (such as source line
numbers) using \lst"\special" that most DVI previewers do not recognize
and may complain about.

\item
{\ActiveDVI} does not currently recognizes virtual fonts, but \lst"dvicopy" 
can be used to expand them. See the option \lst"-dvicopy" in Section~\ref
{sec/dvicopy}. 

\end {itemize}

\subsection{Defining your own previewer}

To use your own command as a previewer, you must choose either type
\lst"-dvi" or \lst"-ps" . In particular, your previewer should 
accept \lst"SIGUSR1" (for \lst"-dvi") signal or \lst"SIGHUP" (for \lst"-ps") 
signal and respond by reloading the file.

\subsection{Viewing with acroread}

This does not work because they is no simple way to tell
\lst"acroread" to reload its file in batch. 

\section{Implementation principles}

In short, {\sc \whizzy} is selecting a small slice of the document that 
you are editing around the cursor (according to the selected mode) 
and redisplay the slice incrementally as it changes through edition. 
\begin {itemize}

\item {\bf Emacs is watching you} typing and moving in the 
Emacs buffer attached to the {\LaTeX} source file that your editing and keeps
saving the current slice (current slide, section, or subsection, according
to the mode).

\item {\bf A shell-script daemon}
keeps recompiling whenever a new slice (or other files) are produced, and if
recompilation succeeds, tels the previewer to updates the display of the slice.

\item {\bf A few {\LaTeX} macros} allow to build a specialized
format with all macro loaded, which considerably speed up the time for
slicing. Additionally, the slice is a bit instrumented to show the cursor,
and includes specials that allows back-pointing from the DVI file into the
Emacs buffer.

\end {itemize}
The rest of this section briefly describe these three parts\footnote {This
section is not quite up-to-date, hence it puts emphasis on the original
design, but several aspects have changed significantly since the first
version. Implementation of more recent features is thus omitted.}, and
their interactions.


\subsection {Emacs code}

The main trick is to use \lst"post-command-hook" to make Emacs watch
changes.  It uses \lst"buffer-modified-tick" to tell if any editing has
actually occurred, and compare the point position with the (remembered)
position of the region being displayed to see if saving should occur.  It
uses \lst"sit-for" to delay slicing until at least the time of slice
computation has ellapsed since last saving, a significant number of editing
changes has occurred, or iddleness.

{\whizzy} can also display the cursor position, in which case slices are
also recomputed when the cursor moves, but with lower priority.

\subsection {{\LaTeX} code}

The main TeX trick is to build a format specialized to the current
document  so as to avoid reloading the
whole macros at each compilation. This is (almost\footnote{{\tt
$\backslash$begin\{document\}} should be typed as such without any white
white space}) entirely transparent, that is, the source file does not have
to understand this trick.

This is implemented by redefining \lst"\documentclass" which in turn
redefines \lst"\document" to execute \lst"\dump" (after redefining
\lst"\document" to its old value and \lst"\documentclass" so that it skips
everything till \lst"\document"). This is robust ---and seems 
to work with rather complex macros. 

The specialized format can be used in two modes: by default it expects a
full document: it them dumps counters at sectioning commands (chapters, 
sections, and subsections). This is useful to correctly 
numberred sections and pages on slices. 

There are also a a few other used to get more advanced behavior, especially
to dump source line numbers and file names so that the previewer can
transform clicks into source file positions. 

When building the format, {\whizzy} also look for a local file of name
\lst"whizzy.sty", which if existing is loaded at the end of the macros. 
This may be used to add other macros in {whizzy} mode, {\em e.g.} 
some {\TeX} environments may be redefined to changed they type setting,
according to whether the current line is inside or outside the environment. 
(We have written such an extension for an exercise package that sends the
answers at the end in an appendix, unless the cursor is inside the answer,
in which case the answer is in-lined.)

\subsection {Bash code}

There is no real trick there. This is a shell-script watching the pool
(a directory where slices and other new version of files must be dropped). 
It them recompiles a slice and wait for input (in stdin). 
It recognizes a few one-line commands as input {\tt reformat}, {\tt
dupplex}, and by default just watch for the presence of a new slice. 
It recompiles the format file (and the page and section number, but in batch
mode) whenever the source file (its Unix date) has changed  and 
recompiles the slice whenever it is present (since {\whizzy} renames ---hence
removes--- the slice before processing it).

If the file has been recompiled successfully, it triggers the previewer
(ghostscript or xdvi) so that it rereads the dvi or ps file. Otherwise, it
processes the {\TeX} log file and tries to locate the error. It then sends part
of the log file with annotations to the \lst"*TeX-shell*" buffer from which
Emacs has been {\whizzy}, so that Emacs can report the error. 

\subsection {Interaction between the components} 

The control is normally done by Emacs, which launches and kills the Unix
daemon. Quitting the previewer should be noticed by the daemon, which tells
Emacs to turn mode off before exiting. 

Muliple {\whizzy} running on the same file would certainly raise racing
conditions between files and would not make sense. 
For that purpose, the daemon pid is saved in a file and {\whizzy}
will kill any old {\whizzy} process on startup. 

\section {\WhizzyEdit ing}


When used together with Active-DVI, {\whizzy} could be made much mode
powerful. In particular, it would be quite easy to to lift {\whizzy} from an
incremental viewer to an assistant editor.

Active-DVI could easily provide a notion of active boxes.
These would be recognized by
\lst"\special" annotations preceeding  boxes. 
Active boxes would be autoraise on focus and could be moved or resized with
the mouse. Rather than displaying actions on screen, which would be unaware
of {\TeX} position stategies, actions should rather be reported in stdout,
as is already done for positions. 

Active-DVI actions could then easily be interepreted by {\whizzy} by
adjusting or inserting the correct vertical or horizontal dimension around
active objects, and processed in the next slice. Thanks to the short
incremental loop, this would (almost) appear as if actions where executed by
Active-DVI.

\paragraph {Experimental (this may be changed in firture versions)}

{\em

This dream is becoming real...
%
Some mouse editing, such as repositioning and resizing of {\tt boxes}, {\tt
minipages}, {\tt glue} and a few drawing commands can already be done in
{\whizzy} ---provided you have a recent version of Active-DVI.

See the example {\tt example/edit/main.tex} and the package
{\tt example/edit/whizzedit.sty} in the distribution.

}


\end{document}

- Clicks

- Say that cursor does not always appear.
% LocalWords:  whizzy advi html whizzytex HeVeA BGCOLOR Didier INRIA ee ou ing
% LocalWords:  ghostview xdvi DVI XEmacs Redhat tgz untar gunzip xf usr config
% LocalWords:  tex minibuffer FMT online dvi dview args autoload ps pre mv
% LocalWords:  BASENAME hyperref gv HUP pid dvips xvf cd wget WhizzyTeX WYSIWIG
% LocalWords:  RET preprocess basename fmt hugelatex BASICFILENAME SIGHUP recog
% LocalWords:  SIGUSR nizes acroread localmacros ghostscript
