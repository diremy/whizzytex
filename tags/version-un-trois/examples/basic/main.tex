%; whizzy  -dvi

\documentclass{article}

\begin{document}

This directory also contains a configuration file \texttt{whizzy.el} which
will be automatically loaded.  However, local configuration, {i.e.} the above
line, takes priority declarations in file \texttt{whizzy.el}.  To see this,
you may the line above (or simply add a white space in front of it), and
observe that the declaration in \texttt{whizzy.el} will indeed be taken into
account.

This directory also contains a file \texttt{whizzy.sty}, which is
automatically loaded before executing \verb$\begin{document}$.

This directory also contains a configuration file \texttt{whizzy.sh}.
This file will be loaded every time WhizzyTeX is run. To see this, look 
at the \texttt{initialization} log file (you may use the menu entry 
\texttt{View log}). 

Usually, you do not need any of these configuration files. 

You may try the menu entry \texttt{Customize slice} and enter 
\verb"\Large". The latex command \verb"\Large" will then be inserted
automatically at the beginning of every slice, and in this case let your
document be typeset in bigger font. To cancel it, just let the slice 
customozation be the empty string. 

\section {First section}

Some text in the section.   
 
\subsection {Sub}
  
Some text in the subsection.

\subsubsection {SubSub} 

Some text in the sub-subsection. 


\section{Foo}

\subsection {Bar}

\paragraph {Par}

And a paragraph aaa a jjj

\end{document}   

